\documentclass[11pt]{amsart}
\usepackage[utf8]{inputenc}
\usepackage[T1]{fontenc}
\usepackage[english]{babel}
\usepackage[a4paper,margin=1in]{geometry}
\usepackage{amsmath,amssymb,amsthm}
\usepackage{mathtools}
\usepackage[protrusion=true,expansion=false]{microtype}
\usepackage{hyperref}
\usepackage{cleveref}
\usepackage{todonotes}

\newtheorem{theorem}{Theorem}[section]
\newtheorem{lemma}[theorem]{Lemma}
\newtheorem{corollary}[theorem]{Corollary}
\newtheorem{definition}[theorem]{Definition}
\newtheorem{claim}[theorem]{Claim}
\newtheorem{observation}[theorem]{Obervation}

\DeclareMathOperator{\degen}{degen}
\DeclareMathOperator{\WReach}{WReach}
\DeclareMathOperator{\dom}{dom}
\DeclareMathOperator{\dist}{dist}

\newcommand{\eps}{\varepsilon}
\newcommand{\F}{\mathcal{F}}
\newcommand{\G}{\mathcal{G}}
\newcommand{\Z}{\mathbb{Z}}
\newcommand{\Oh}{\mathcal{O}}
\newcommand{\val}{\mathrm{val}}

\newcommand{\wninline}[2][inline]{\todo[color=green!80,#1]{{\textbf{W:} #2}}}
\newcommand{\bbinline}[2][inline]{\todo[color=blue!50,#1]{{\textbf{B:} #2}}}
\newcommand{\azinline}[2][inline]{\todo[color=red!80,#1]{{\textbf{A:} #2}}}
\newcommand{\mpinline}[2][inline]{\todo[color=yellow!50,#1]{{\textbf{M:} #2}}}

\title[Fully Dominating Set in Graphs of Bounded Degeneracy]{Fully Dynamic Randomized Data Structure for Approximate Dominating Set in Graphs of Bounded Degeneracy}

\author{Bartłomiej Bosek}
\address{Institute of Theoretical Computer Science, Jagiellonian University in Kraków, \and Institute of Informatics, University of Warsaw, Poland}
\email{bartlomiej.bosek@protonmail.com}

\author{Wojciech Nadara}
\address{Institute of Informatics, University of Warsaw, Poland}
\email{w.nadara@mimuw.edu.pl}

\author{Michał Pilipczuk}
\address{Institute of Informatics, University of Warsaw, Poland}
\email{michal.pilipczuk@mimuw.edu.pl}

\author{Anna Zych-Pawlewicz}
\address{Institute of Informatics, University of Warsaw, Poland}
\email{anka@mimuw.edu.pl}

\thanks{This work is a~part of project BOBR that has received funding from the European Research Council (ERC) under the European Union's Horizon 2020 research and innovation programme (grant agreement No. 948057).}

\keywords{Dynamic data structures, bounded degeneracy, dominating set, hypergraph matching}

\subjclass[2020]{68P05, 68W27, 05C85}

\begin{document}

\maketitle

\begin{abstract}
We present the first fully dynamic data structure maintaining a constant approximation of the minimum dominating set in graphs of bounded degeneracy.
For a dynamic graph with degeneracy bounded by $\alpha$, our structure maintains a dominating set of size at most $(c \alpha + 1)^2 \cdot \gamma(G_t)$, where $\gamma(G_t)$ is the size of the minimum dominating set, and $c$ is a constant.
Edge insertion and deletion operations take expected amortized time $O(\alpha^2 \log n)$, and the structure occupies $O(\alpha n + m)$ space.
The key technique is the reduction of the problem to maintaining a maximal matching in a hypergraph of low rank, combined with a dynamic acyclic orientation.
\end{abstract}

\section{Introduction}

The minimum dominating set problem is one of the fundamental problems in graph theory.
In the dynamic version, where the graph evolves through edge insertions and deletions, efficiently maintaining an approximate solution remains a challenge.
For general graphs, strong lower bounds on time complexity are known. Therefore, it is natural to consider classes of graphs with special structure.
We focus on graphs of bounded degeneracy, a class containing planar graphs, graphs of bounded genus, and graphs with excluded minors.
For graphs with constant degeneracy $\alpha$, efficient static approximation algorithms are known \cite{LenzenWattenhofer10}.
We present the first fully dynamic data structure for this problem.

\paragraph{Main Result.} We construct a fully dynamic data structure that, for graphs with degeneracy at most $\alpha$, maintains a dominating set of size at most $(c \alpha + 1)^2 \cdot \gamma(G_t)$ with expected amortized update time $O(\alpha^2 \log n)$ and space complexity $O(\alpha n + m)$.

\paragraph{Techniques.} We utilize three components: a dynamic acyclic orientation with bounded out-degree \cite{BrodalFagerberg99}, a reduction to maximal matching in a hypergraph, and a dynamic matching algorithm \cite{AssadiSolomon21}.
The first component relies on the result by Brodal and Fagerberg from 1999 regarding dynamic representations of sparse graphs.
The second utilizes combinatorial properties of reachable sets, generalizing the approach of Lenzen and Wattenhofer \cite{LenzenWattenhofer10}.
The third uses the algorithm by Assadi and Solomon from 2021 for dynamic matching in hypergraphs.

\section{Preliminaries}

\subsection{Graphs and Domination}

Let $G = (V, E)$ denote an undirected graph.
We denote the neighbors of a vertex $v$ by $N_G(v)$, and its closed neighborhood by $N_G[v] = \{v\} \cup N_G(v)$.
The distance between vertices $u, v$ is denoted by $\dist_G(u,v)$.

\begin{definition}[Domination]
For $r \in \mathbb{N}$, a set $D \subseteq V$ is an \emph{$r$-dominating set} in graph $G$ if every vertex $v \in V$ is at a distance of at most $r$ from some vertex in $D$, i.e., $\min_{d \in D} \dist_G(v,d) \leq r$.
For $r=1$, we simply speak of a dominating set. The size of the smallest $r$-dominating set is denoted by $\dom_r(G)$. In particular, $\gamma(G) = \dom_1(G)$.
\end{definition}

\subsection{Degeneracy}

\begin{definition}[Degeneracy]
A graph $G$ is \emph{$d$-degenerate} if every non-empty induced subgraph $H \subseteq G$ contains a vertex of degree at most $d$.
The \emph{degeneracy} of a graph $G$ is the smallest $d$ for which $G$ is $d$-degenerate, denoted by $\degen(G)$.
\end{definition}

Equivalently, $\degen(G)$ is the smallest $d$ for which there exists a linear order $\sigma: V \to \{1,\ldots,n\}$ such that every vertex $v$ has at most $d$ neighbors $u$ with $\sigma(u) < \sigma(v)$ (the so-called left degree).
The maximum left degree in order $\sigma$ is denoted by $\Delta^{\text{left}}_\sigma$.

\subsection{Weak Reachable Sets}

\begin{definition}[Weak $r$-reachable sets]
\label{def:wreach}
For a graph $G$, an order $\sigma$, and a vertex $v$, we define the \emph{weak backward $r$-reachable set} as
\[
\WReach_r[G,\sigma,v] = \{ u \in V \mid \exists \text{ path } u = p_0, \ldots, p_\ell = v \text{ with } \ell \leq r \text{ such that } \sigma(u) \leq \sigma(p_i) \text{ for } i=1,\ldots,\ell \}.
\]
The \emph{weak $r$-coloring number} of the order $\sigma$ is defined as
$\text{wcol}_r(G,\sigma) = \max_{v \in V} |\WReach_r[G,\sigma,v]|$.
The weak $r$-coloring number of the graph is
$\text{wcol}_r(G) = \min_\sigma \text{wcol}_r(G,\sigma)$.
\end{definition}

Note that for $r=1$, we have $\WReach_1[G,\sigma,v] = \{v\} \cup \{u \in N_G(v) \mid \sigma(u) < \sigma(v)\}$, so $\text{wcol}_1(G,\sigma) = 1 + \Delta^{\text{left}}_\sigma$.

\subsection{Hypergraphs}

\begin{definition}[Hypergraph and Matching]
A \emph{hypergraph} over a vertex set $V$ is a family of subsets $\mathcal{H} \subseteq 2^V$.
A \emph{matching} in $\mathcal{H}$ is a subfamily $\mathcal{M} \subseteq \mathcal{H}$ of pairwise disjoint sets.
A matching $\mathcal{M}$ is \emph{maximal with respect to inclusion} if there is no $S \in \mathcal{H} \setminus \mathcal{M}$ disjoint from all sets in $\mathcal{M}$.
\end{definition}

\section{Combinatorics of Weak Reachable Sets}\label{sec:kombinatoryka}

In this section, we present key combinatorial results connecting weak reachable sets with $r$-dominance.
All theorems are formulated for arbitrary $r \in \mathbb{N}$; however, in the dynamic construction, we will apply them only for $r=1$.

\subsection{Domination via Matching}

\begin{lemma}[Domination]
\label{lem:domination}
Let $G$ be a graph, $\sigma$ an order on $V(G)$, and $r \in \mathbb{N}$.
Consider the family
$\mathcal{F}_r(\sigma) = \{\WReach_r[G,\sigma,v] \mid v \in V\}$.
If $\mathcal{M} \subseteq \mathcal{F}_r(\sigma)$ is a matching maximal with respect to inclusion, then the set $D = \bigcup_{X \in \mathcal{M}} X$ is an $r$-dominating set in $G$.
\end{lemma}

\begin{proof}
Take any $x \in V$. If $x \in D$, then $x$ is $r$-dominated by itself (since $\dist_G(x,x) = 0 \leq r$).
Assume $x \notin D$. Consider the set $X_x = \WReach_r[G,\sigma,x] \in \mathcal{F}_r(\sigma)$.
Since $\mathcal{M}$ is maximal, $X_x$ cannot be disjoint from all sets in $\mathcal{M}$.
Therefore, there exists $Y \in \mathcal{M}$ such that $X_x \cap Y \neq \emptyset$.

Choose $d \in X_x \cap Y$.
By the definition of $X_x = \WReach_r[G,\sigma,x]$, there exists a path $P: d = p_0, p_1, \ldots, p_\ell = x$ of length $\ell \leq r$ such that $\sigma(d) \leq \sigma(p_i)$ for $i=1,\ldots,\ell$.
In particular, $\dist_G(x,d) \leq \ell \leq r$. Since $d \in Y \subseteq D$, vertex $x$ is $r$-dominated by $d \in D$.
\end{proof}

\subsection{Transversal of the Optimal Solution}

\begin{lemma}[Transversal]
\label{lem:transversal}
Let $D^*$ be an optimal $r$-dominating set in $G$, i.e., $|D^*| = \dom_r(G)$.
Let $\sigma$ be an order on $V$ and $\mathcal{F}_r(\sigma) = \{\WReach_r[G,\sigma,v] \mid v \in V\}$.
The transversal
\[
T = \bigcup_{d \in D^*} \WReach_r[G,\sigma,d]
\]
has the property that every set in $\mathcal{F}_r(\sigma)$ intersects $T$.
\end{lemma}

\begin{proof}
Take any $v \in V$ and $X_v = \WReach_r[G,\sigma,v] \in \mathcal{F}_r(\sigma)$.
Since $D^*$ is an $r$-dominating set, there exists $d \in D^*$ such that $\dist_G(v,d) \leq r$.
Let $P: v = p_0, p_1, \ldots, p_k = d$ be a shortest path, where $k \leq r$.
Let $u$ be the vertex on $P$ with the smallest $\sigma(u)$.

The fragment of path $P$ from $u$ to $v$ has length at most $k \leq r$ and all vertices on this fragment (except $u$) have indices at least $\sigma(u)$, so $u \in \WReach_r[G,\sigma,v] = X_v$.
Analogously, the fragment from $u$ to $d$ implies $u \in \WReach_r[G,\sigma,d] \subseteq T$.
Therefore, $u \in X_v \cap T$, so $X_v \cap T \neq \emptyset$.
\end{proof}

\subsection{Approximation via Matching}

\begin{theorem}[Static Approximation]
\label{thm:approx-static}
Let $G$ be a graph, $\sigma$ an order on $V$, and $r \in \mathbb{N}$.
Consider $\mathcal{F}_r(\sigma) = \{\WReach_r[G,\sigma,v] \mid v \in V\}$. If $\mathcal{M} \subseteq \mathcal{F}_r(\sigma)$ is a matching maximal with respect to inclusion, then the set $D = \bigcup_{X \in \mathcal{M}} X$ is an $r$-dominating set in $G$ and
\[
|D| \leq (1 + \Delta^{\text{left}}_\sigma)^2 \cdot \dom_r(G).
\]
In particular, for $r=1$: $|D| \leq \text{wcol}_1(G,\sigma)^2 \cdot \gamma(G)$.
\end{theorem}

\begin{proof}
From Lemma~\ref{lem:domination}, we know that $D$ is an $r$-dominating set. It remains to estimate $|D|$.

Let $D^* \subseteq V$ be an optimal $r$-dominating set, i.e., $|D^*| = \dom_r(G)$. By Lemma~\ref{lem:transversal}, the transversal $T = \bigcup_{d \in D^*} \WReach_r[G,\sigma,d]$ intersects every set in $\mathcal{F}_r(\sigma)$.
Since $\mathcal{M}$ is a matching, the sets in $\mathcal{M}$ are pairwise disjoint. Every set $X \in \mathcal{M}$ must intersect $T$ (by Lemma~\ref{lem:transversal}).
Therefore,
\[
|\mathcal{M}| \leq |T| = \left| \bigcup_{d \in D^*} \WReach_r[G,\sigma,d] \right| \leq \sum_{d \in D^*} |\WReach_r[G,\sigma,d]| \leq |D^*| \cdot \text{wcol}_r(G,\sigma).
\]

Each set $X \in \mathcal{M}$ has size at most $\text{wcol}_r(G,\sigma)$ (by the definition of the weak coloring number). Thus,
\[
|D| = \left| \bigcup_{X \in \mathcal{M}} X \right| \leq \sum_{X \in \mathcal{M}} |X| \leq |\mathcal{M}| \cdot \text{wcol}_r(G,\sigma) \leq |D^*| \cdot \text{wcol}_r(G,\sigma)^2 = \dom_r(G) \cdot \text{wcol}_r(G,\sigma)^2.
\]

For $r=1$, we have $\text{wcol}_1(G,\sigma) = 1 + \Delta^{\text{left}}_\sigma$, which yields the desired inequality.
\end{proof}

\begin{corollary}[Approximation for small weak coloring number]
\label{cor:approx-wcol}
If $\text{wcol}_r(G,\sigma) \leq c$, then there exists an $r$-dominating set $D$ in $G$ of size $|D| \leq c^2 \cdot \dom_r(G)$.
\end{corollary}

\section{Data Structure Interface Specification}

We present the interface of a fully dynamic data structure maintaining an approximation of the minimum dominating set in graphs with degeneracy bounded by $\alpha$.

\subsection{Operations}

The data structure provides the following operations:

\begin{description}
\item[\textsc{Init}()] Initializes an empty structure for a graph on $n$ vertices with no edges.
Effect: The structure is ready to accept \textsc{InsertEdge} and \textsc{DeleteEdge} operations. Time complexity: $O(1)$.

\item[\textsc{InsertEdge}$(u,v)$] Adds the edge $\{u,v\}$ to the graph (if it does not exist).
Effect: Graph $G_t$ contains the edge $\{u,v\}$, and the structure maintains a dominating set $D_t$ for $G_t$.
Time complexity: expected amortized $O(\alpha^2 \log n)$.

\item[\textsc{DeleteEdge}$(u,v)$] Removes the edge $\{u,v\}$ from the graph (if it exists).
Effect: Graph $G_t$ does not contain the edge $\{u,v\}$, and the structure maintains a dominating set $D_t$ for $G_t$.
Time complexity: expected amortized $O(\alpha^2 \log n)$.

\item[\textsc{QueryDominatingSet}()] Returns the current dominating set $D_t$.
Effect: The returned set $D_t$ satisfies $|D_t| \leq (c\alpha + 1)^2 \cdot \gamma(G_t)$, where $c$ is an absolute constant. Time complexity: $O(|D_t|) = O(\alpha^2 \gamma(G_t))$.
\end{description}

\subsection{Guarantees}

The data structure guarantees the following properties:

\begin{itemize}
\item \textbf{Correctness:} At any moment $t$, the returned set $D_t$ is a dominating set of the graph $G_t$.
\item \textbf{Approximation:} The size $|D_t|$ is at most $(c\alpha + 1)^2 \cdot \gamma(G_t)$, where $c$ is the constant from the Brodal-Fagerberg algorithm.
\item \textbf{Time Complexity:} Operations \textsc{InsertEdge} and \textsc{DeleteEdge} run in expected amortized time $O(\alpha^2 \log n)$ against a non-adaptive adversary.
\item \textbf{Space Complexity:} The structure occupies $O(\alpha n + m_t)$ space, where $m_t = |E_t|$.
\end{itemize}

\section{Data Structure Description}

The structure consists of three components storing the following information:

\subsection{Orientation Component (Brodal-Fagerberg)}

Maintains an acyclic orientation $\vec{G}_t$ of the graph $G_t$ with maximum out-degree $\Delta^+_{\max} \leq c\alpha$.
Stored data:
\begin{itemize}
\item For each edge $\{u,v\} \in E_t$: its orientation (either $(u,v)$ or $(v,u)$ in $\vec{G}_t$).
\item For each vertex $v$: the set of successors $N^+_{\vec{G}_t}(v) = \{u \mid (v,u) \in \vec{G}_t\}$.
\end{itemize}
Space: $O(n + m_t)$ (adjacency list for the directed graph).

\subsection{Hypergraph Component}

For the current orientation $\vec{G}_t$, it maintains a hypergraph $\mathcal{H}_t = \{S_t(v) \mid v \in V\}$, where $S_t(v) = \{v\} \cup N^+_{\vec{G}_t}(v)$.
Stored data:
\begin{itemize}
\item For each vertex $v$: the set $S_t(v)$ (implemented as a dynamic list).
\item Mapping from vertex $v$ to all sets $S_t(u)$ containing $v$ (for efficient updates).
\end{itemize}
Space: Each set $S_t(v)$ has size at most $1 + c\alpha$. We have $n$ sets, so $O(\alpha n)$ elements in total.

\subsection{Matching Component (Assadi-Solomon)}

Maintains a maximal matching $\mathcal{M}_t \subseteq \mathcal{H}_t$. Stored data as described in \cite{AssadiSolomon21}, Theorem 1.1.
Space: $O(r \cdot n + n) = O(\alpha n)$ for a hypergraph with $n$ hyperedges of rank $r = O(\alpha)$.

\subsection{Dominating Set}

The current dominating set $D_t = \bigcup_{X \in \mathcal{M}_t} X$ is not explicitly stored as a separate structure.
Instead, it is computed on demand in the \textsc{QueryDominatingSet}() operation.

Space: No additional space (the set is computed dynamically).

\subsection{Memory Summary}

Total space: $O(n + m_t) + O(\alpha n) + O(\alpha n) = O(\alpha n + m_t)$.

\section{Implementation of Procedures}

\subsection{Procedure \textsc{Init}}

Creates an empty structure for a graph on $n$ vertices with no edges.
\paragraph{Implementation.} Initializes components: empty adjacency lists for orientation (time $O(n)$, but can be reduced to $O(1)$ via lazy creation), empty hypergraph with $n$ single-element sets $S(v) = \{v\}$ (time $O(n)$, but analogously can be $O(1)$), initialization of the Assadi-Solomon structure for an empty hypergraph ($O(1)$).
\paragraph{Complexity.} With lazy implementation: $O(1)$.

\subsection{Procedure \textsc{InsertEdge}$(u,v)$}

Adds the edge $\{u,v\}$ to the graph.

\paragraph{Implementation.} Calls the edge addition operation in the Brodal-Fagerberg component.
This component deterministically decides on the orientation of the new edge (e.g., $(u,v)$) and may perform a sequence of orientation flips of other edges to maintain acyclicity and the constraint $\Delta^+_{\max} \leq c\alpha$.
Each flip of an edge $\{x,y\}$ (changing from $(x,y)$ to $(y,x)$) generates two modifications to the hypergraph: removing $y$ from the set $S(x)$ and adding $x$ to the set $S(y)$.
These modifications are passed to the Assadi-Solomon component, which updates the maximal matching.

\paragraph{Complexity.} The Brodal-Fagerberg algorithm (Theorem 2 in \cite{BrodalFagerberg99}) guarantees an expected amortized number of flips of $O(\log n)$ per operation.
Each flip corresponds to two modifications of a hypergraph of rank $r = O(\alpha)$.
The Assadi-Solomon algorithm (Theorem 1.1 in \cite{AssadiSolomon21}) has an expected amortized time of $O(r^2)$ per hypergraph modification against a non-adaptive adversary.
In total: $O(\log n) \cdot O(\alpha^2) = O(\alpha^2 \log n)$.

\subsection{Procedure \textsc{DeleteEdge}$(u,v)$}

Removes the edge $\{u,v\}$ from the graph.
\paragraph{Implementation.} Analogous to \textsc{InsertEdge}: calls the operation in the Brodal-Fagerberg component, which may perform $O(\log n)$ flips, each generating two hypergraph modifications.
\paragraph{Complexity.} $O(\alpha^2 \log n)$ expected amortized time (identical analysis as for \textsc{InsertEdge}).

\subsection{Procedure \textsc{QueryDominatingSet}()}

Returns the current dominating set.
\paragraph{Implementation.} Iterates through all sets $X \in \mathcal{M}_t$ in the matching and collects the union $D_t = \bigcup_{X \in \mathcal{M}_t} X$ (eliminating duplicates, e.g., via a hash set).
\paragraph{Complexity.} $|\mathcal{M}_t| \leq c\alpha \cdot \gamma(G_t)$ (from the proof of Theorem~\ref{thm:approx-static}), each set has size $O(\alpha)$, so the total number of elements is $O(\alpha^2 \gamma(G_t))$, hence the time is $O(|D_t|) = O(\alpha^2 \gamma(G_t))$.

\section{Proofs of Correctness and Complexity}

\subsection{Correctness}

\begin{theorem}[Correctness]
After any sequence of operations, the structure maintains a set $D_t$ which is a dominating set of the graph $G_t$ with size $|D_t| \leq (c\alpha + 1)^2 \cdot \gamma(G_t)$, where $c$ is the constant from the Brodal-Fagerberg algorithm.
\end{theorem}

\begin{proof}
Let $\vec{G}_t$ be the acyclic orientation maintained by the Brodal-Fagerberg component. From Theorem 2 in \cite{BrodalFagerberg99} (specifically citing equation (2) on page 345), we know that $\Delta^+_{\max}(\vec{G}_t) \leq c \cdot \degen(G_t) \leq c \alpha$, where $c$ is an absolute constant from that algorithm.
Since the orientation is acyclic, there exists a reverse topological order $\sigma_t$. For each vertex $v$, we define $S_t(v) = \{v\} \cup N^+_{\vec{G}_t}(v)$.
By construction, the hypergraph $\mathcal{H}_t = \{S_t(v) \mid v \in V\}$ satisfies $S_t(v) = \WReach_1[G_t, \sigma_t, v]$ for every $v$ (proof analogous to Lemma~\ref{lem:domination} in the preliminaries section but for orientation).
The Assadi-Solomon component maintains a matching $\mathcal{M}_t \subseteq \mathcal{H}_t$ maximal with respect to inclusion.
From Lemma~\ref{lem:domination} applied for $r=1$, we obtain that $D_t = \bigcup_{X \in \mathcal{M}_t} X$ is a dominating set in $G_t$.
From Theorem~\ref{thm:approx-static} for $r=1$, we have
\[
|D_t| \leq \text{wcol}_1(G_t, \sigma_t)^2 \cdot \gamma(G_t) = (1 + \Delta^{\text{left}}_{\sigma_t})^2 \cdot \gamma(G_t).
\]
Since $\Delta^{\text{left}}_{\sigma_t} = \Delta^+_{\max}(\vec{G}_t) \leq c\alpha$ (from the fact that $\sigma_t$ is a reverse topological order), we have
\[
|D_t| \leq (1 + c\alpha)^2 \cdot \gamma(G_t) = (c\alpha + 1)^2 \cdot \gamma(G_t).
\qedhere
\]
\end{proof}

\subsection{Time Complexity}

\begin{theorem}[Time Complexity]
Operations \textsc{InsertEdge} and \textsc{DeleteEdge} take expected amortized time $O(\alpha^2 \log n)$ against a non-adaptive adversary.
\end{theorem}

\begin{proof}
The Brodal-Fagerberg algorithm (Theorem 2 in \cite{BrodalFagerberg99}, specifically Corollary 1 on page 349) guarantees that an edge operation generates an expected amortized number of $O(\log n)$ orientation flips in a graph with $n$ vertices.
Each flip of an edge $\{x,y\}$ results in two hypergraph modifications: removing an element from one set and adding it to another.
The size of each set in the hypergraph is at most $r = 1 + c\alpha = O(\alpha)$ (where $c$ is the constant from the Brodal-Fagerberg algorithm).
The Assadi-Solomon algorithm (Theorem 1.1 in \cite{AssadiSolomon21}, specifically point (a) on page 2204) guarantees expected amortized time $O(r^2)$ per operation of adding or removing an element in a hyperedge of rank $r$, against a non-adaptive adversary.
For our hypergraph $r = O(\alpha)$, so each modification costs $O(\alpha^2)$.
Crucially, the sequence of hypergraph modifications is non-adaptive with respect to the randomness of the Assadi-Solomon algorithm.
The flips are deterministically determined by the Brodal-Fagerberg algorithm based on the sequence of graph operations, which is assumed to be fixed in advance (non-adaptive adversary).
Therefore, the sequence of hypergraph modifications does not depend on the random choices within the Assadi-Solomon structure.
In total, one graph operation generates $O(\log n)$ flips, each costing $O(\alpha^2)$, yielding an expected amortized time of $O(\alpha^2 \log n)$ per operation.
\end{proof}

\subsection{Space Complexity}

\begin{theorem}[Space]
The structure occupies $O(\alpha n + m_t)$ space.
\end{theorem}

\begin{proof}
The Brodal-Fagerberg component stores a directed graph with $n$ vertices and $m_t$ edges, which requires $O(n + m_t)$ space (adjacency lists).
The hypergraph consists of $n$ sets, each of size at most $1 + c\alpha = O(\alpha)$.
The total number of elements is $O(\alpha n)$. Additional structures (e.g., reverse pointers) do not increase the asymptotics.
The Assadi-Solomon structure (Theorem 1.1 in \cite{AssadiSolomon21}, point (b)) for a hypergraph with $n$ hyperedges of rank $r$ occupies $O(rn + n) = O(rn)$ space.
For $r = O(\alpha)$, we have $O(\alpha n)$.

Total: $O(n + m_t) + O(\alpha n) + O(\alpha n) = O(\alpha n + m_t)$.
\end{proof}

\section{Summary and Open Questions}

We have presented the first fully dynamic data structure maintaining a constant approximation of the minimum dominating set in graphs of bounded degeneracy.
Key to this are: dynamic acyclic orientation \cite{BrodalFagerberg99}, reduction to maximal matching in a hypergraph, and a dynamic matching algorithm \cite{AssadiSolomon21}.
The result generalizes the static approach of Lenzen and Wattenhofer \cite{LenzenWattenhofer10} to the fully dynamic model.

\paragraph{Open Question.} The presented results concern $1$-domination (standard domination). In Section~\ref{sec:kombinatoryka} (Combinatorics of Weak Reachable Sets), we showed that the combinatorics of weak reachable sets generalizes to arbitrary $r \geq 1$.
The main obstacle to extending the dynamic construction to $r$-domination for $r > 1$ is the lack of an efficient method for maintaining an order $\sigma$ with a small value of $\text{wcol}_r(G,\sigma)$ for $r > 1$.
For $r=1$, we used acyclic orientation, but for larger $r$, more complex structures are needed.
Is there a fully dynamic data structure that, for a fixed $r > 1$ and graphs of bounded degeneracy, maintains an order $\sigma$ with $\text{wcol}_r(G,\sigma) = O(1)$ in polylogarithmic time per operation?
Solving this problem would open the way to efficient dynamic approximation of $r$-domination and many other distance-based problems in sparse graphs.

\section{Bezładne początki tekstu Wojtka}

\begin{lemma}
	Let $f$ be a function that takes two arguments --- a graph $G$ and its vertex $v$, such that $f(G, v) \ge 0$ for all valid pairs of arguments. Suppose that for any fixed function $g : V(G) \to \Z$, there exists a data structure $R_{f, g}(G)$ that for a dynamic graph $G$ is able to report $\sum_{v \in V(G)} f(G, v)g(v)$ after each update, and a data structure $V_{f}(G)$ that for a dynamic graph $G$ and for a given vertex $v$ is able to check if $f(G, v)>0$. Then, there exists a randomized data structure $P_{f}(G)$ with no false positives that is able to report any $v$ such that $f(G, v) > 0$, or that no such vertex exists. If the amortized time and space complexities of $R$ and $V$ per a single update or query are $\Oh(T_R)$, $\Oh(S_R)$, $\Oh(T_V)$ and $\Oh(S_V)$, respectively, then the amortized time and space complexities of $P$ are $\Oh((T_R+T_V) \log |V(G)| \cdot \log \frac{1}{\eps})$ and $\Oh((S_R+S_V) \log |V(G)| \cdot \log \frac{1}{\eps})$, where $\eps$ is the probability of a false negative.
\end{lemma}	

\wninline{Przerzucic poniższe do prelimsów?}
\begin{definition}
	Let $G$ be a graph and $v \in V(G), v' \not\in V(G)$. The operation of \textit{adding a pendant} $vv'$ produces a graph $G'$ such that $V(G') = V(G) \cup \{v'\}$ and $E(G') = E(G) \cup \{vv'\}$.
\end{definition}

\begin{observation} \label{obs:pendants-bnd-exp}
	Let $\G$ be a class of graphs with bounded expansion and let $\G'$ be a class of graphs that can be obtained from $\G$ by adding pendants arbitrarily many times. Then $\G'$ is also of bounded expansion.
\end{observation}

\wninline{textit się nie wyróżnia w definicjach?}
\begin{definition}
	We say that $\F$ is a \textit{type of mappings} if it is a function such that for undirected graphs $H, G$ with colored edges $\F(H, G)$ returns a subset of all mappings from $V(H)$ to $V(G)$.
\end{definition}

	For example, homomorphisms, subgraph isomorphisms and induced subgraph isomorphisms are all types of mappings. We shall denote them by $\mathcal{HOM}$, $\mathcal{SUB}$ and $\mathcal{ISUB}$, i.e. $\mathcal{HOM}(H, G)$, $\mathcal{SUB}(H, G)$ and $\mathcal{ISUB}(H, G)$ will denote the set of all homomorphisms, subgraph isomorphisms and induced subgraph isomorphisms, respectively.
	
\begin{definition}
	Let $\F$ be a type of mappings and let $\F(H, G, v, u)$ for $v \in V(H)$ and $u \in V(G)$ be the restriction of $F(H, G)$ to the mappings $\phi$ such that $\phi(v)=u$.
	We say that $\F$ is \textit{reasonable} if and only if it satisfies the following property: Let $H$ and $G$ be undirected graphs with colored edges, where $H$ does not have isolated vertices. Let $v \in V(H), u \in V(G)$, $H'$ be a graph $H$ with added pendant $vv'$ and $G'$ be a graph with added pendant $uu'$, where $vv'$ and $uu'$ have the same color that is not present in $H$ and $G$. Then, $\F(H', G') = \F(H', G', v', u')$ and there is a natural bijection between $\F(H', G')$ and $\F(H, G, v, u)$ given by restricting any $\phi' \in F(H', G')$ to  $V(H)$. 
\end{definition}

\begin{claim}
	Homomorphisms, subgraph isomorphisms and induced subgraph isomorphisms are reasonable types of mappings.
\end{claim}
\begin{proof}
	Let $\F \in \{\mathcal{HOM}, \mathcal{SUB}, \mathcal{ISUB}\}$. Let $\phi \in \F(H', G')$. All three of these types are homomorphisms, so we have that $\phi$ is a homomorphism from $H$ to $G$. As $vv'$ and $uu'$ have equal colors that are different from colors of any other edges in $H'$ and $G'$ we must have that either $\phi(v)=u, \phi(v')=u'$ or $\phi(v)=u', \phi(v')=u$. However, as $v$ is not isolated in $H$, it has an adjacent edge with a different color than $vv'$, so it is not possible that $\phi(v)=u'$. Hence $\F(H', G') = \F(H', G', v', u') = \F(H', G', v, u)$. One can then readily verify that restricting any member of $\F(H', G')$ to $V(H)$ defines a bijection between $\F(H', G')$ and $\F(H, G, v, u)$ for any $\F$ from $\{\mathcal{HOM}, \mathcal{SUB}, \mathcal{ISUB}\}$.
\end{proof}


\begin{lemma} Let us assume that there exists a randomized data structure $S_{H, v, k, \mathcal{F}}(G)$ solving the following problem: Let $H$ be a graph without isolated vertices and with edges colored with $\{1, \ldots, k\}$, $v$ be a vertex of $H$, $G$ be a dynamic graph that belongs to a fixed class of graphs $\mathcal{G}$ of bounded expansion. Let $\mathcal{F}$ be a reasonable type of mappings. Then, the data structure is able to either provide a vertex $u \in V(G)$ such that there exists $\phi$ such that $\phi(v)=u$ and $\phi \in \F(H, G)$, or state that such $\phi$ does not exist. Any update to $G$ is handled in amortized $f_{\mathcal{G}}(|V(H)|, k) \log^{g(|V(H)|, k)}(|V(G)|)$ time complexity (for some monotonic functions $f_{\mathcal{G}}$ and $g$) and the queries never provide false positives, but may provide false negatives with probability at most $\eps$.
	
Then, there exists a data structure $P_{H, k, \mathcal{F}}(G)$ solving an analogous problem, which provides $\phi$ such that $\phi \in \F(H, G)$, or state that it does not exist. Any update to $G$ is handled in amortized $f'_{\mathcal{G}}(|V(H)|, k) \log^{g'(|V(H)|, k)}(|V(G)|+|V(H)|)$ time complexity for some functions $f_{\mathcal{G}}'$ and $g'$. The queries never provide false negatives, but may provide false negatives with probability at most $|V(H)| \eps$.

%\wninline{Dopisac zalozenie, ze dodanie pendanta do G zachowuje bycie w $\mathcal{G}$.}
\end{lemma}
\begin{proof}
	Let $c=|V(H)|$ and let $V(H) = \{v_1, \ldots, v_c\}$. Let us create a sequence of graphs $H_0, \ldots, H_{c}$, where $H=H_0$ and $H_{i}$ is created from $H_{i-1}$ by adding a $v_ip_i$ pendant colored with the color $k+i$. 
	
	Let $S_i \coloneqq S_{H_i, v_{i+1}, k + i, \mathcal{F}}$ for $i=0, \ldots, c-1$. Any update to $G$ is passed to all of $S_0, \ldots, S_{c-1}$. The data structure $S_i$ will be used to identify the image of $v_{i+1}$ after already fixing images of $v_1, \ldots, v_i$.
	That is, let us assume that we have already identified vertices $u_1, \ldots, u_i \in V(G)$ such that there exists a mapping $\phi \in \F(H, G)$, where $\phi(v_j)=u_j$ for $j \le i$. Let $G_0, G_1, \ldots, G_i$ be the sequence of graphs such that $G=G_0$ and $G_i$ is created from $G_{i-1}$ by adding a $u_iq_i$ pendant colored with the color $k+i$. Let us turn $S_i(G)$ into $S_i(G_i)$ by adding the corresponding $i$ pendants. Let $\G'$ be the class of graphs obtained from $\G$ by adding arbitrarily many pendants. Based on \ref{obs:pendants-bnd-exp} we have that $\G'$ is of bounded expansion and $G_i \in \G'$.
	
	By the assumption that $\F$ is reasonable and by induction on $i$, it is clear that there is a natural bijection between mappings $\phi' \in \F(H_i, G_i)$ and mappings $\phi in \F(H, G)$ such that $\phi(v_i)=u_i$, where that bijection fulfills $\phi'|_{V(H)} = \phi$. As we assumed that there exists $\phi \in \F(H, G)$ such that $\phi(v_i) = u_i$, we conclude that $\F(H_i, G_i)$ is nonempty. Hence, $S_i$ is able to provide a vertex $u_{i+1}$ such that there exists $\phi' \in \F(H_i, G_i)$ where $\phi'(v_{i+1})=u_{i+1}$. Moreover it has to be the case that $u_{i+1} \in V(G)$ (that is, it is not possible that $u_{i+1}=q_j$ for some $j\le i$), so by restricting $\phi'$ to $V(H)$ we get a mapping $\phi$ such that $\phi(v_j)=u_j$ for $j\le i+1$. By repeating this reasoning for $i=0, \ldots, c-1$, we get a full mapping $\phi \in \F(H, G)$, as desired.
	
	Note that we can set $f'_{\G}(|V(H)|, k) = |V(H)|^2 f_{\G'}(2|V(H)|, k+|V(H)|)$ and $g'(|V(H)|, k) = g(2|V(H)|, k+|V(H)|)$ and $|V(G_i)| \le |V(G)| + |V(H)|$. As we make $|V(H)|$ queries to all $S_i$, the tota probability of not succeeding at some step is at most $|V(H)|\eps$.
	
\end{proof}

\begin{definition}
	Let $H$ and $G$ be graphs and $f$ be a function $f : V(H) \times V(G) \to \Z$. Let $\phi : V(H) \to V(G)$ be a mapping. Then, by the \textit{value of} $\phi$, denoted as $\val_f(\phi)$, we denote the following value $\val_f(\phi) = \prod_{v \in V(H)} f(v, \phi(v))$. 
\end{definition}

In \cite{DvorakTumaWADS} Dvor{\'{a}}k and Tuma prove the following: 

\begin{theorem}
	Let $H$ be a fixed graph, $\G$ be a class of graphs and $G \in \G$ be a dynamic graph. 
\end{theorem}



\bibliographystyle{plain}
\bibliography{references}

\end{document}