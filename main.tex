\documentclass[11pt]{article}
\usepackage[margin=1in,bottom=1in]{geometry}
\usepackage[utf8]{inputenc}
\usepackage[T1]{fontenc}
\usepackage[english]{babel}

\usepackage{authblk}
\usepackage{tabularx}
\usepackage{mdframed}
\usepackage{libertine}
\usepackage{amsthm}
\usepackage{amsmath}
\DeclareMathOperator{\symdiff}{\triangle}
\usepackage{amssymb}
\usepackage{stmaryrd}
\usepackage{mathrsfs} 
\usepackage{mathtools}
\usepackage{thm-restate}
\usepackage[colorlinks = true,linkcolor = blue,urlcolor = blue, citecolor = blue]{hyperref}
\usepackage[capitalize, nameinlink]{cleveref}
\usepackage{url}
\usepackage{todonotes}
\usepackage{mathrsfs} 
\usepackage{soul}
\usepackage{relsize}
%\usepackage{enumerate}
\usepackage{enumitem}
\usepackage[all,defaultlines=3]{nowidow}
\usepackage[mathlines]{lineno}
\usepackage{multicol}
\usepackage{enumitem}
\usepackage{xspace}
\usepackage{lineno}
\usepackage{nicefrac}
\usepackage{wasysym}
\usepackage{comment}
\usepackage{bbold}
\linenumbers

\usepackage{textpos}
\setlist[itemize]{topsep=4pt,itemsep=3pt,parsep=0pt} 
\setlist[enumerate]{topsep=4pt,itemsep=3pt,parsep=0pt} 
 %\linenumbers

\crefname{claim}{Claim}{Claims}
\crefname{figure}{Figure}{Figures}
\renewcommand{\preceq}{\preccurlyeq}
\renewcommand{\succeq}{\succcurlyeq}
\renewcommand{\subset}{\subseteq}
\newtheorem{theorem}{Theorem}[section]

\newtheorem{corollary}[theorem]{Corollary}
\newtheorem{conjecture}{Conjecture}
\newtheorem{remark}{Remark}
\newtheorem{observation}[theorem]{Observation}
\newtheorem{lemma}[theorem]{Lemma}
\newtheorem{case}{Case}
\newtheorem{claim}[theorem]{Claim}
\newtheorem{fact}[theorem]{Fact}
\newtheorem{proposition}[theorem]{Proposition}
\theoremstyle{definition}
\newtheorem{definition}[theorem]{Definition}
\theoremstyle{plain}
\newtheorem{property}{Property}
\theoremstyle{definition}
\newtheorem{example}{Example}
\newtheorem*{example*}{Example}

\numberwithin{equation}{section}

\renewcommand{\cal}{\mathcal}

\newenvironment{claimproof}[1][Proof of the claim.]{%
  \renewcommand{\qedsymbol}{$\lrcorner$}%
  \begin{proof}[#1]%
}{%
  \end{proof}%
}

\newcommand{\cc}{\mathsf{cc}}
\newcommand{\bd}{\partial}

\newcommand{\bn}{\mathsf{bn}}
\newcommand{\tw}{\mathsf{tw}}
\newcommand{\wtw}{\mathsf{wtw}}
\newcommand{\bw}{\mathsf{bw}}
\newcommand{\wbw}{\mathsf{wbw}}

\newcommand{\wei}{\mathbf{w}}

\newcommand{\Pp}{\mathcal{P}}
\newcommand{\Qq}{\mathcal{Q}}
\newcommand{\Ww}{\mathcal{W}}

\newcommand{\tup}[1]{{\bar{#1}}}

\newcommand{\Prob}{\mathbb{P}}
\newcommand{\Exp}{\mathbb{E}}


\newcommand{\clos}{\mathrm{cl}}
\newcommand{\bnd}{\partial}


\def\phi{\varphi}
\newcommand{\N}{\mathbb{N}}
\newcommand{\R}{\mathbb{R}}
\newcommand{\Z}{\mathbb{Z}}
\newcommand{\Zp}{\Z^+}
\newcommand{\wh}{\widehat}

\newcommand{\Leaves}{\mathrm{Leaves}}
\newcommand{\Ball}{\mathrm{Ball}}
\newcommand{\dist}{\mathrm{dist}}
\newcommand{\cw}{\mathsf{cvw}}
\newcommand{\wcw}{\mathsf{wcvw}}
\newcommand{\ct}{\mathsf{cut}}
\newcommand{\md}{\mathsf{mid}}

\newcommand{\Aa}{\mathcal{A}}
\newcommand{\Bb}{\mathcal{B}}
\newcommand{\Ee}{\mathcal{E}}

\newcommand{\Tt}{\mathcal{T}}
\newcommand{\Ll}{\mathcal{L}}
\newcommand{\Ff}{\mathcal{F}}
\newcommand{\Ss}{\mathcal{S}}

\newcommand{\F}{\mathcal{F}}
\newcommand{\G}{\mathcal{G}}
\newcommand{\val}{\mathrm{val}}

\newcommand{\from}{\colon}
\newcommand{\Cc}{\mathscr C}
\newcommand{\Dd}{\mathscr D}
\def\epsilon{\varepsilon}
\def\eps{\varepsilon}
\newcommand{\Oh}{\mathcal{O}}

\newcommand{\id}{\mathrm{id}}
\renewcommand{\emptyset}{\varnothing}

\newcommand{\michal}[1]{\todo[color=pink!40]{Micha\l{}: #1}}
\newcommand{\michalin}[1]{\todo[size=\normalsize,inline,color=pink!40]{Micha\l{}: #1}}
\newcommand{\wninline}[2][inline]{\todo[color=green!50,#1]{{\textbf{W:} #2}}}
\newcommand{\bbinline}[2][inline]{\todo[color=blue!50,#1]{{\textbf{B:} #2}}}
\newcommand{\azinline}[2][inline]{\todo[color=yellow!50,#1]{{\textbf{A:} #2}}}
\newcommand{\mpinline}[2][inline]{\todo[color=pink!40,#1]{{\textbf{M:} #2}}}


\renewcommand{\leq}{\leqslant}
\renewcommand{\geq}{\geqslant}
\renewcommand{\le}{\leq}
\renewcommand{\ge}{\geq}

\renewcommand{\setminus}{-}
\newcommand{\set}[1]{\{#1\}}
\newcommand{\setof}[2]{\{#1 \colon#2\}}

\newcommand{\cnst}[3]{{#1}_{#2}^{\ref{#3}}}

\newcommand{\aff}[1]{\normalsize\textcolor{black!60}{#1}}

\usepackage{tikz}

\begin{document}

\newcommand{\funding}{All the authors were supported by the project BOBR that received funding from the European Research Council (ERC) under the European Union’s Horizon 2020 research and innovation programme, grant agreement No. 948057.}


%\title{Minor Containment and Disjoint Paths in almost-linear time\thanks{\funding}}
\title{Dynamic domination and independence in sparse graphs\footnote{\funding}}
%\date{\today}
\date{}
	\author[1,2]{Bart\l{}omiej Bosek}
	\author[1]{Wojciech Nadara}
	\author[1]{Micha\l{} Pilipczuk}
	\author[1]{Anna Zych-Pawlewicz}
	\affil[1]{\aff{University of Warsaw, Poland}}
	\affil[2]{\aff{Jagiellonian University, Cracow, Poland}}
\maketitle

\begin{abstract}
	Let $\Cc$ be a class of graphs of bounded expansion and $d,k\in \N$ be fixed. We give a dynamic data structure that for a given dynamic graph $G$, updated by edge insertions and deletions subject to the promise that $G\in \Cc$ at all times, maintains the answer to the following two queries:
	\begin{itemize}[nosep]
		\item Does $G$ contain a distance-$d$ dominating set of size $k$?
		\item Does $G$ contain a distance-$d$ independent set of size $k$?
	\end{itemize}
	The data structure is randomized, of Las Vegas type.
	The expected amortized update time is $\log^c n$, where $n$ is the vertex count of $G$ and $c$ is a constant that depends only on $d$, $k$, and $\Cc$. In case of the first query, the data structure can also output a distance-$d$ dominating set of size $k$, if existent.
	
	We also prove that when $d=1$, our data structure for the dominating set query can be implemented even if we only assume that the maintained graph $G$ has arboricity bounded by a constant $\alpha$, yielding a simpler data structure and an improved update time of $f(k,\alpha)\cdot \log^3 n$ (still expected and amortized). Finally, we prove that in graphs of arboricity at most $\alpha$, one can maintain a constant-factor approximation of the minimum size of a (distance-$1$) dominating set with amortized update time $TODO$.
\end{abstract}

% \begin{textblock}{20}(-1.75, 3.6)
% \includegraphics[width=40px]{logo-erc}%
% \end{textblock}
% \begin{textblock}{20}(-1.75, 4.6)
% \includegraphics[width=40px]{logo-eu}%
% \end{textblock}

\newpage

\clearpage
\setcounter{page}{1}

\section{Introduction}\label{sec:intro}

\wninline{Wojtek}
\azinline{Ania}
\bbinline{Bartek}
\michalin{Michał1}
\mpinline{Michał2}
\section{Preliminaries}\label{sec:prelims}

\wninline{Będę chciał mieć definicję co znaczy ``data structure $S(G)$ for a dynamic graph $G$'', aby nie pisać tego 20 razy}

\begin{definition}
	Let $G$ be a graph and $v \in V(G), v' \not\in V(G)$. The operation of \textit{adding a pendant} $vv'$ produces a graph $G'$ such that $V(G') = V(G) \cup \{v'\}$ and $E(G') = E(G) \cup \{vv'\}$.
\end{definition}

\begin{observation} \label{obs:pendants-bnd-exp}
	Let $\G$ be a class of graphs with bounded expansion and let $\G'$ be a class of graphs that can be obtained from $\G$ by adding pendants arbitrarily many times. Then $\G'$ is also of bounded expansion.
\end{observation}
\wninline{Czy powyższe wymaga dowodu?}

\begin{definition}
	We say that $\F$ is a \textit{type of mappings} if it is a function such that for undirected graphs $H, G$ with colored edges $\F(H, G)$ returns a subset of all mappings from $V(H)$ to $V(G)$.
\end{definition}

For example, homomorphisms, subgraph isomorphisms and induced subgraph isomorphisms are all types of mappings. We shall denote them by $\mathcal{HOM}$, $\mathcal{SUB}$ and $\mathcal{ISUB}$, i.e. $\mathcal{HOM}(H, G)$, $\mathcal{SUB}(H, G)$ and $\mathcal{ISUB}(H, G)$ will denote the set of all homomorphisms, subgraph isomorphisms and induced subgraph isomorphisms, respectively.
\section{(Semi-)ladder algorithms for domination and independence}\label{sec:ladders}

Here we recall \cite{FabianskiPST19}. The goal is to reduce the task of designing dynamic data structures on a class $\Cc$ to three ingredients:
\begin{itemize}
	\item Proof that graphs from $\Cc$ do not admit long (semi-)ladders.
	\item Implementation of the oracle: Given a set $D$ with $|D|\leq k$, find $v$ satisfying $\dist(v,D)>d$.
	\item Implementation of the oracle: Given a set $D$ with $|D|\leq k$, find $v$ satisfying $\dist(u,v)\leq d$ for each $u\in D$.
\end{itemize}
We should name those oracles somehow.
\section{Counting and finding small patterns in sparse graphs}\label{sec:dvorak}

Here we recall and extend relevant stuff from Dvo\v{r}\'ak and T\r{u}ma~\cite{DvorakT13}.

The main result of the work of Dvo\v{r}\'ak and T\r{u}ma~\cite{DvorakT13} can be expressed as following:

\begin{theorem} \cite{DvorakT13} \label{thm:dvorak-tuma}
	Let $H$ be a fixed graph, $\G$ be a class of graphs and $G \in \G$ be a dynamic graph, where edges of $H$ and $G$ are colored with colors $\{1, \ldots, k\}$. Let $\F$ be either $\mathcal{HOM}$, $\mathcal{SUB}$, or $\mathcal{ISUB}$. Then, there exists a data structure $C_{\F, H}(G)$, which is able to determine $|\F(H, G)|$ after each update. If $\G$ is a class of bounded expansion, then the data structure processes each update in $\Oh(\log^h |V(G)|)$, where $h = {|V(H)| \choose 2} - 1$ and its space complexity is $\Oh(|V(G)|)$. If $\G$ is a nowhere-dense class, then each update takes $\Oh(|V(G)|^\eps$ time and its space complexity is $\Oh(|V(G)|^{1 + \eps})$. The $\Oh$ notation hides factors dependent on $\G, H$ and $k$.
\end{theorem}

However, they concluded that even though they are able to count appearances of $H$ in $G$ as an induced subgraph (that is, determine $|\mathcal{ISUB}(H, G)|$), it is not easy to restore any example of such mapping from their data structure as it involves inclusion-exclusion principle. Despite not written explicitly, it can be expected that the authors would have similar troubles restoring examples of subgraph isomorphisms (as the reduction from counting subgraph isomorphisms to counting homomorphisms requires subtractions as well), though they would have likely known how to restore examples of homomorphisms, as that part of the algorithm does not require any subtractions and the restoration of an example can be done through tracing transitions with nonzero contributions in the designed dynamic programming. 

In this work, we are going to resolve these open questions and prove the following:

\begin{theorem} \label{thm:dvorak-tuma-examples}
	Let $H$ be a fixed graph, $\G$ be a class of graphs and $G \in \G$ be a dynamic graph, where edges of $H$ and $G$ are colored with colors $\{1, \ldots, k\}$. Let $\F$ be either $\mathcal{HOM}$, $\mathcal{SUB}$, or $\mathcal{ISUB}$. Let $\delta \in (0, 1)$ be a real number. Then, there exists a randomized data structure $E_{\F, H}(G)$, which after each update is able to report that either $\F(H, G)$ is empty, or provide some $\phi \in \F(H, G)|$. If $\G$ is a class of bounded expansion, then the data structure processes each update in $\Oh(\log^{h+1} |V(G)| \log{\frac{1}{\delta}})$, where $h = {|V(H)| \choose 2} - 1$ and its space complexity is $\Oh(|V(G)| \log |V(G)| \log {\frac{1}{\delta}})$. If $\G$ is a nowhere-dense class, then each update takes $\Oh(|V(G)|^\eps \log{\frac{1}{\delta}})$ time and its space complexity is $\Oh(|V(G)|^{1 + \eps} \log{\frac{1}{\delta}})$. The $\Oh$ notation hides factors dependent on $\G, H$ and $k$. The data structure never provide false positives, but may fail to provide $\phi$ with probability at most $\delta$.
\end{theorem}

For the algorithms in the following sections we will only need the case $\F = \mathcal{HOM}$, which was the only case that was easy to handle for \cite{DvorakT13} (despite not being stated and proven), but the toolbox that we need to introduce for our main result is of great help in resolving the harder cases of $\F = \mathcal{SUB}$ and $\F = \mathcal{ISUB}$ that can be viewed as interesting results of independent interest.

\wninline{@Bartek, napiszesz dowód tego lematu? De facto już to zrobiłeś w dom-one i trzeba by go tylko przystosować do języka tutaj}

The following lemma, which draws an inspiration from \cite{Majewski24} and \cite{Nadara22}, is a handy tool that helps in restorations of examples while given only counting functions.
\begin{lemma}
	Let $f$ be a function that takes two arguments --- a graph $G$ and its vertex $v$, such that $f(G, v) \ge 0$ for all valid pairs of arguments.
	
	Suppose that for any fixed weighting function $w : V(G) \to \Z$, there exists a data structure $R_{f, w}(G)$ that for a dynamic graph $G$ is able to report $\sum_{v \in V(G)} f(G, v)w(v)$ after each update.
	
	Also, suppose that either:
	\begin{enumerate}
		\item \label{it:binary-f} $f(G, v) \in \{0, 1\}$ for all valid arguments, or
		\item there exists a data structure $V_{f}(G)$ that for a dynamic graph $G$ and for a given vertex $v$ is able to check if $f(G, v)>0$.
	\end{enumerate}
	Then, there exists a randomized data structure $P_{f}(G)$ with no false positives that is able to report any $v$ such that $f(G, v) > 0$, or that no such vertex exists. If the amortized time and space complexities of $R$ and $V$ per a single update or query are $\Oh(T_R)$, $\Oh(S_R)$, $\Oh(T_V)$ and $\Oh(S_V)$, respectively, then the amortized time and space complexities of $P$ are $\Oh((T_R+T_V) \cdot \log |V(G)| \cdot \log \frac{1}{\eps})$ and $\Oh((S_R+S_V) \cdot \log |V(G)| \cdot \log \frac{1}{\eps})$, where $\eps$ is the probability of a false negative and where we assume that $T_V = S_V = 0$ in the case \cref{it:binary-f}, where $V$ is not required.
\end{lemma}	


\begin{definition}
	Let $\F$ be a type of mappings and let $\F(H, G, v, u)$ for $v \in V(H)$ and $u \in V(G)$ be the restriction of $F(H, G)$ to the mappings $\phi$ such that $\phi(v)=u$.
	We say that $\F$ is \textit{reasonable} if and only if it satisfies the following property: Let $H$ and $G$ be undirected graphs with colored edges, where $H$ does not have isolated vertices. Let $v \in V(H), u \in V(G)$, $H'$ be a graph $H$ with added pendant $vv'$ and $G'$ be a graph $G$ with added pendant $uu'$, where $vv'$ and $uu'$ have the same color that is not present in $H$ and $G$. Then, $\F(H', G') = \F(H', G', v', u')$ and there is a natural bijection between $\F(H', G')$ and $\F(H, G, v, u)$ given by restricting any $\phi' \in F(H', G')$ to  $V(H)$. 
\end{definition}

\begin{claim}
	Homomorphisms, subgraph isomorphisms and induced subgraph isomorphisms are reasonable types of mappings.
\end{claim}
\begin{proof}
	Let $\F \in \{\mathcal{HOM}, \mathcal{SUB}, \mathcal{ISUB}\}$. Let $\phi \in \F(H', G')$. All three of these types are homomorphisms, so we have that $\phi$ is a homomorphism from $H$ to $G$. As $vv'$ and $uu'$ have equal colors that are different from colors of any other edges in $H'$ and $G'$ we must have that either $\phi(v)=u, \phi(v')=u'$ or $\phi(v)=u', \phi(v')=u$. However, as $v$ is not isolated in $H$, it has an adjacent edge with a different color than $vv'$, so it is not possible that $\phi(v)=u'$. Hence $\F(H', G') = \F(H', G', v', u') = \F(H', G', v, u)$. One can then readily verify that restricting any member of $\F(H', G')$ to $V(H)$ defines a bijection between $\F(H', G')$ and $\F(H, G, v, u)$ for any $\F$ from $\{\mathcal{HOM}, \mathcal{SUB}, \mathcal{ISUB}\}$.
\end{proof}


\begin{lemma} Let us assume that there exists a randomized data structure $S_{H, v, k, \mathcal{F}}(G)$ solving the following problem: Let $H$ be a graph without isolated vertices and with edges colored with $\{1, \ldots, k\}$, $v$ be a vertex of $H$, $G$ be a dynamic graph that belongs to a fixed class of graphs $\mathcal{G}$ of bounded expansion. Let $\mathcal{F}$ be a reasonable type of mappings. Then, the data structure is able to either provide a vertex $u \in V(G)$ such that there exists $\phi$ such that $\phi(v)=u$ and $\phi \in \F(H, G)$, or state that such $\phi$ does not exist. Any update to $G$ is handled in amortized $f_{\mathcal{G}}(|V(H)|, k) \log^{g(|V(H)|, k)}(|V(G)|)$ time complexity (for some monotonic functions $f_{\mathcal{G}}$ and $g$) and the queries never provide false positives, but may provide false negatives with probability at most $\eps$.
	
	Then, there exists a data structure $P_{H, k, \mathcal{F}}(G)$ solving an analogous problem, which provides $\phi$ such that $\phi \in \F(H, G)$, or state that it does not exist. Any update to $G$ is handled in amortized $f'_{\mathcal{G}}(|V(H)|, k) \log^{g'(|V(H)|, k)}(|V(G)|+|V(H)|)$ time complexity for some functions $f_{\mathcal{G}}'$ and $g'$. The queries never provide false negatives, but may provide false negatives with probability at most $|V(H)| \eps$.
	
	%\wninline{Dopisac zalozenie, ze dodanie pendanta do G zachowuje bycie w $\mathcal{G}$.}
\end{lemma}
\begin{proof}
	Let $c=|V(H)|$ and let $V(H) = \{v_1, \ldots, v_c\}$. Let us create a sequence of graphs $H_0, \ldots, H_{c}$, where $H=H_0$ and $H_{i}$ is created from $H_{i-1}$ by adding a $v_ip_i$ pendant colored with the color $k+i$. 
	
	Let $S_i \coloneqq S_{H_i, v_{i+1}, k + i, \mathcal{F}}$ for $i=0, \ldots, c-1$. Any update to $G$ is passed to all of $S_0, \ldots, S_{c-1}$. The data structure $S_i$ will be used to identify the image of $v_{i+1}$ after already fixing images of $v_1, \ldots, v_i$.
	That is, let us assume that we have already identified vertices $u_1, \ldots, u_i \in V(G)$ such that there exists a mapping $\phi \in \F(H, G)$, where $\phi(v_j)=u_j$ for $j \le i$. Let $G_0, G_1, \ldots, G_i$ be the sequence of graphs such that $G=G_0$ and $G_i$ is created from $G_{i-1}$ by adding a $u_iq_i$ pendant colored with the color $k+i$. Let us turn $S_i(G)$ into $S_i(G_i)$ by adding the corresponding $i$ pendants. Let $\G'$ be the class of graphs obtained from $\G$ by adding arbitrarily many pendants. Based on \cref{obs:pendants-bnd-exp} we have that $\G'$ is of bounded expansion and $G_i \in \G'$.
	
	By the assumption that $\F$ is reasonable and by induction on $i$, it is clear that there is a natural bijection between mappings $\phi' \in \F(H_i, G_i)$ and mappings $\phi \in \F(H, G)$ such that $\phi(v_i)=u_i$, where that bijection fulfills $\phi'|_{V(H)} = \phi$. As we assumed that there exists $\phi \in \F(H, G)$ such that $\phi(v_i) = u_i$, we conclude that $\F(H_i, G_i)$ is nonempty. Hence, $S_i$ is able to provide a vertex $u_{i+1}$ such that there exists $\phi' \in \F(H_i, G_i)$ where $\phi'(v_{i+1})=u_{i+1}$. Moreover it has to be the case that $u_{i+1} \in V(G)$ (that is, it is not possible that $u_{i+1}=q_j$ for some $j\le i$), so by restricting $\phi'$ to $V(H)$ we get a mapping $\phi$ such that $\phi(v_j)=u_j$ for $j\le i+1$. By repeating this reasoning for $i=0, \ldots, c-1$, we get a full mapping $\phi \in \F(H, G)$, as desired.
	
	Note that we can set $f'_{\G}(|V(H)|, k) = |V(H)|^2 f_{\G'}(2|V(H)|, k+|V(H)|)$ and $g'(|V(H)|, k) = g(2|V(H)|, k+|V(H)|)$ and $|V(G_i)| \le |V(G)| + |V(H)|$. As we make $|V(H)|$ queries to all $S_i$, the total probability of not succeeding at some step is at most $|V(H)|\eps$.
	
\end{proof}

\begin{definition}
	Let $H$ and $G$ be graphs and $w$ be a weighting function $w : V(H) \times V(G) \to \Z$. Let $\phi : V(H) \to V(G)$ be a mapping. Then, by the \textit{value of} $\phi$, denoted as $\val_w(\phi)$, we denote the following value $\val_w(\phi) = \prod_{v \in V(H)} f(v, \phi(v))$. Consequently, for a type of mappings $\F$, we are going to define $\val_w(\F(H, G))$ as $\sum_{\phi \in \F(H, G)} \val_w(\phi)$.
\end{definition}

We are now going to present a weighted version of \cref{thm:dvorak-tuma}:

\begin{theorem} \label{lem:weighted-dvorak}
	Let $H$ be a fixed graph, $\G$ be a class of graphs and $G \in \G$ be a dynamic graph, where edges of $H$ and $G$ are colored with colors $\{1, \ldots, k\}$ and $w$ be a weighting function $w : V(H) \times V(G) \to \Z$. Let $\F$ be either $\mathcal{HOM}$, $\mathcal{SUB}$, or $\mathcal{ISUB}$. Then, there exists a data structure $W_{\F, H, w}(G)$, which is able to determine $\val_w(\F(H, G))$ after each update. If $\G$ is a class of bounded expansion, then the data structure processes each update in $\Oh(\log^h |V(G)|)$, where $h = {|V(H)| \choose 2} - 1$ and its space complexity is $\Oh(|V(G)|)$. If $\G$ is a nowhere-dense class, then each update takes $\Oh(|V(G)|^\eps$ time and its space complexity is $\Oh(|V(G)|^{1 + \eps})$. The $\Oh$ notation hides factors dependent on $\G, H$ and $k$.
\end{theorem}

On a high level, \cref{thm:dvorak-tuma} is proven in a way that counting induced subgraph isomorphisms is reduced to counting subgraph isomorphisms, which is reduced to counting homomorphisms, which is reduced to counting homomorphisms in a special case, where both $H$ and $G$ are additionally directed, but $H$ is a so called \textit{elder graph}. The last case is solved through a dynamic programming routine. As can be expected, incorporating weights into a dynamic programming routine can be easily done and having that, weighted versions are easily pulled back to homomorphisms, subgraph isomorphisms and induced subgraph isomorphism. We defer more detailed proof of \cref{lem:weighted-dvorak} to \cref{sec:app-weighted-dvorak}.

\wninline{Muszę jakoś ogarnąć narrację w kwestii braku tych izolowanych wierzchołków}

\section{Implementation of oracles in classes of bounded expansion}\label{sec:implementation}

Here we implement the oracles for an arbitrary distance $d$ in classes of bounded expansion.
\section{Distance-$1$ domination}

Here we give the data structure for distance-$1$ dominating sets in graphs of bounded arboricity. This boils down to arguing the three ingredients from \cref{sec:ladders}: no long semi-ladders and implementation of both oracles.
\section{Dynamic approximation of distance-$1$ dominating sets}

Here we give a dynamic constant-factor approximation for the domination number in graphs of bounded arboricity.
\section{Conclusions}

Here we ask questions:
\begin{itemize}
	\item Dynamic $\mathsf{FO}$ model-checking on bounded expansion.
	\item Dynamic approximation for $d>1$ on bounded expansion.  
\end{itemize}

\bibliographystyle{alpha}
\bibliography{references}

\clearpage
\appendix
\section{The proof of \cref{lem:weighted-dvorak}} \label{sec:app-weighted-dvorak}
In Theorem 14 of \cite{DvorakTumaArxiv} Dvo\v{r}\'ak and T\r{u}ma prove that for a fixed directed elder graph $H$ and dynamic directed graph $G$ with maximum in-degree at most $D$, where both have edges colored by $\{0, 1, \ldots, k\}$, it is possible to maintain the number of homomorphisms from $H$ to $G$ in $\Oh(f(|V(H)|, D))$ time complexity per update. A directed graph $H$ is called an \textit{elder graph} if it has the property that $(u, w), (v, w) \in E(H)$ imply that $u$ and $v$ are adjacent as well, that is, either $(u, v) \in E(H)$ or $(v, u) \in E(H)$.

We generalize their statement to the following weighted version:

\begin{lemma}
	Let $D$ be an integer, $H$ be a fixed directed elder graph, $G$ be a dynamic directed graph with maximum in-degree at most $D$, where both have edges colored by $\{0, 1, \ldots, k\}$ and $w$ be a weighting function $w : V(H) \times V(G) \to \Z$. There exists a data structure $AHom_{H, k, D, w}(G)$ that maintains $\val_w(\mathcal{HOM}(H, G))$. Each update to the data structure takes $f(|V(H)|, D)$ time complexity and the space complexity of it is $\Oh(f(|V(H)|, D) \cdot |V(G)|)$ for some function $f$. 
\end{lemma}

\begin{proof}
	The generalization to the weighted version requires just a few minor modifications to the proof of the original Theorem 14 of \cite{DvorakTumaArxiv}, therefore we are only going to highlight the differences between them and refer the reader to the original result for a more detailed version. We are also going to use the terminology of that article (that is, notions like \textit{vineyard, clan, extended clan, ghost}).
	
	During the initialization of the data structure we fix any vineyard $T$ for $H$. For a clan $C$ and vertices $v, w_1, \ldots, w_m \in V(G)$, by $\mathcal{HOM}_{H, T}(C, v, w_1, \ldots, w_m, G)$ we are going to denote the set of homomorphisms from an extended clan $C^*$ to $G$ such that $r(C)$ is mapped to $v$ and ghosts $g_1, \ldots, g_m$ of $C$ are mapped to $w_1, \ldots, w_m$. A value of a partial homomorphism $\phi$ from $H$ to $G$ belonging to that set is defined naturally as the product of $w(x_i, y_i)$ over all $x_i \in V(C^*)$, where $\phi(x_i) = y_i$, and the value of a whole set is defined as the sum of values of its elements.
	
	For each clan $C \neq V(H)$ with $m$ ghosts and each $m$-tuple of vertices $w_1, \ldots, w_m$ of $G$ in the unweighted version the number $S(C, w_1, \ldots, w_m) = \sum_{v \in N_1^=(w_1)} |\mathcal{HOM}_{H, T}(C, v, w_1, \ldots, w_m, G)|$ was recorded, however we will record the number $S_w(C, w_1, \ldots, w_m) = \sum_{v \in N_1^=(w_1)} \val_w(\mathcal{HOM}_{H, T}(C, v, w_1, \ldots, w_m, G))$ instead. 
			
	The original proof expresses the difference of the sets $\mathcal{HOM}_{H, T}(C, v, w_1, \ldots, w_m, G)$ before and after an addition of an edge $e$ to $G$ as a disjoint sum of some sets, where each of them results from a different guess of which edges of $H$ will be mapped to $e$. In each of these cases, a partial homomorphism $\phi$ is determined and the problem of extending it to a homomorphism of full $C^*$ decomposes to a few independent subproblems, hence the set of ways to extend the fixed partial homomorphisms resulting from that guess to a full homomorphisms of $C^*$ can be expressed as a cartesian product of sets of homomorphisms for the smaller instances. If $C_i$ and $g_1^i, \ldots, g_{m_i}^i$ are the clan and the ghosts of the $i$-th subproblem, then the number of ways to extend $\phi$ to a full homomorphisms of $C^*$ can be expressed as $\prod_{i=1}^{t} S(C_i, \phi(g_1^i), \ldots, g_{m_i}^i)$, whereas the incurred contribution to the value of all accounted homomorphisms will be $\val_w(\phi) \cdot \prod_{i=1}^{t}S_w(C_i, \phi(g_1^i), \ldots, g_{m_i}^i)$.
	
	We remark that the calculation above crucially relies on the properties of $\val_w$ that $\val_w(X \sqcup Y) = \val_w(X) + \val_w(Y)$ and $\val_w(P \times Q) = \val_w(P) \cdot \val_w(Q)$, where $P$ and $Q$ are acting on disjoint subsets $V_P$ and $V_Q$ of $V(H)$ and each element $\phi = (\phi_P, \phi_Q) \in P \times Q$ is understood as a unique mapping on $V_P \sqcup V_Q$ such that $\phi|_{V_P} = \phi_P$ and $\phi|_{V_Q} = \phi_Q$.  
	
\end{proof}

\end{document}
