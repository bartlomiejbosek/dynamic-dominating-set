\documentclass[11pt]{article}
\usepackage[utf8]{inputenc}
\usepackage[T1]{fontenc}
\usepackage[a4paper,margin=1in]{geometry}
\usepackage{amsmath,amssymb,amsthm,mathtools}
\usepackage[protrusion=true,expansion=false]{microtype}
\usepackage{hyperref}

\newtheorem{theorem}{Theorem}[section]
\newtheorem{lemma}[theorem]{Lemma}
\newtheorem{claim}[theorem]{Claim}
\newtheorem{corollary}[theorem]{Corollary}
\theoremstyle{definition}
\newtheorem{definition}[theorem]{Definition}

\title{Fully Dynamic Randomized Data Structure for Approximate Dominating Set in Graphs of Bounded Degeneracy}
\author{
Bartłomiej Bosek\thanks{Institute of Informatics, University of Warsaw, Poland. Email: \texttt{bartlomiej.bosek@protonmail.com}} \and
Wojciech Nadara\thanks{Institute of Informatics, University of Warsaw, Poland. Email: \texttt{wnadara@mimuw.edu.pl}} \and
Michał Pilipczuk\thanks{Institute of Informatics, University of Warsaw, Poland. Email: \texttt{michal.pilipczuk@mimuw.edu.pl}} \and
Anna Zych-Pawlewicz\thanks{Institute of Informatics, University of Warsaw, Poland. Email: \texttt{anych@mimuw.edu.pl}}
}
\date{\today{}}

\begin{document}
\maketitle

\begin{abstract}
We present a fully dynamic data structure maintaining a dominating set $D_t$ in an evolving undirected graph $G_t=(V,E_t)$ such that $|D_t|\le (\alpha+1)^2\cdot \gamma(G_t)$, where $\alpha$ is a constant upper bound on the degeneracy of all graph states, and $\gamma(G_t)$ denotes the size of a minimum dominating set. Each edge update has amortized expected cost $O(\alpha^3+\alpha^2\log n)$, while a query returns the maintained set $D_t$ in time proportional to $|D_t|$. The construction is modular: (i) we deterministically maintain an orientation with bounded out-degree using the classical scheme for sparse graphs \cite{BrodalFagerberg99}, which limits local changes per update; (ii) over the family of \emph{out-stars} we build a dynamic hypergraph of rank $r\le \alpha+1$ and maintain a maximal matching in it using the method of Assadi and Solomon \cite{AssadiSolomon21}. We prove independently two combinatorial facts crucial for correctness and size guarantees: (A) for any ordering $\sigma$, the union of sets from a maximal matching in the family $\mathrm{WReach}_1[G,\sigma]$ is a dominating set and has size at most $(\Delta^{\mathrm{left}}_\sigma+1)^2\cdot \gamma(G)$; (B) a completely analogous property holds for the family of out-stars with respect to any orientation, which allows us to relate the size of $D_t$ with the parameter $\Delta^+(G_t)\le \alpha$. The whole structure works against an oblivious adversary, according to the analysis model from \cite{AssadiSolomon21}. We do not use any additional assumptions such as $n\le f(\alpha)$; initialization \textsc{Init} creates an empty graph in time $O(1)$.
\end{abstract}

\section{Introduction}

Maintaining an approximate dominating set in the fully dynamic model is a natural test of maturity for methods on sparse graphs: on one hand, the problem is classically hard to approximate, on the other hand—graphs of finite degeneracy ($\alpha<\infty$) provide rich local structure. In this work we combine a deterministic module for edge orientation in graphs of bounded arboricity (and consequently degeneracy) \cite{BrodalFagerberg99} with a local, randomized scheme for maintaining maximal matching in a rank-$r$ hypergraph \cite{AssadiSolomon21}. Main idea: each vertex $v$ generates a hyperedge $S(v)=\{v\}\cup N^+(v)$ (out-star in the current orientation), and we maintain a maximal matching $\mathcal{M}_t$ in this family. The returned set $D_t$ is the union $\bigcup_{S\in \mathcal{M}_t}S$. We prove that $D_t$ always dominates $G_t$, and moreover $|D_t|\le (1+\Delta^+(G_t))^2\cdot \gamma(G_t)$. Since we maintain $\Delta^+(G_t)=O(\alpha)$, we obtain the desired bound $(\alpha+1)^2$.

\paragraph{Adversary and randomness model.}
We use analysis in the \emph{oblivious adversary} model—the sequence of updates does not depend on the algorithm's coin tosses; this is the standard condition required for time and approximation guarantees in \cite{AssadiSolomon21}. We explicitly note this in the specification and analysis.

\section{Preliminaries and Notation}

We work on simple graphs $G=(V,E)$. For $v\in V$ we denote by $N(v)$ the neighborhood, by $N[v]=N(v)\cup\{v\}$ the closed neighborhood. A set $D\subseteq V$ is \emph{dominating} when $N[v]\cap D\neq\emptyset$ for every $v$. Let $\gamma(G)$ denote the size of a minimum dominating set.

\paragraph{Degeneracy.}
The degeneracy of a graph is the smallest number $\alpha$ such that every induced subgraph has a vertex of degree at most $\alpha$. Equivalently: there exists a linear ordering $\sigma$ on $V$ with maximum \emph{left degree} $\Delta^{\mathrm{left}}_\sigma\le \alpha$, i.e., for each $v$ the number of neighbors $u$ with $\sigma(u)<\sigma(v)$ is at most $\alpha$.

\paragraph{Set families and hypergraphs.}
For a family of sets $\mathcal{F}\subseteq 2^U$, a \emph{matching} is a subfamily of pairwise disjoint sets; a \emph{maximal matching}—in the sense of inclusion; $\nu(\mathcal{F})$—the size of a maximum matching. A \emph{transversal} (or hitting set) is $Y\subseteq U$ intersecting every set in $\mathcal{F}$; $\tau(\mathcal{F})$—the size of the smallest transversal. We have $\nu(\mathcal{F})\le \tau(\mathcal{F})$ in any hypergraph.

\paragraph{$\mathrm{WReach}_1$ and left degree.}
For an ordering $\sigma$ and vertex $v$ we define $L_\sigma(v)=\{u\in N(v): \sigma(u)<\sigma(v)\}$ and $\mathrm{WReach}_1[G,\sigma](v)=\{v\}\cup L_\sigma(v)$. Let $\mathcal{H}_\sigma=\{ \mathrm{WReach}_1[G,\sigma](v): v\in V\}$ and $r_\sigma=\max_v|\mathrm{WReach}_1[G,\sigma](v)|=1+\Delta^{\mathrm{left}}_\sigma$.

\paragraph{Orientations and out-stars.}
Let $\vec{G}$ be an orientation of $G$. We denote $N^+(v)$—outgoing neighbors, $\Delta^+(\vec{G})=\max_v |N^+(v)|$. For $v$ we define $S_{\vec{G}}(v)=\{v\}\cup N^+(v)$. Let $\mathcal{H}_{\mathrm{out}}(\vec{G})=\{S_{\vec{G}}(v):v\in V\}$ and $r_{\mathrm{out}}=1+\Delta^+(\vec{G})$.

\section{Two Key Combinatorial Facts}

First, the result required in point 5 for $\mathrm{WReach}_1$.

\begin{lemma}[Maximality $\Rightarrow$ domination for $\mathrm{WReach}_1$]
\label{lem:max-dominates-wreach}
Let $\mathcal{M}$ be a maximal matching (with respect to inclusion) in the hypergraph $\mathcal{H}_\sigma$. Then the set $D=\bigcup_{X\in \mathcal{M}} X$ is a dominating set in $G$.
\end{lemma}

\begin{proof}
Take any $x\in V$. If there exists $X\in\mathcal{M}$ with $x\in X$, then $x$ is dominated (it even belongs to $D$). If not, consider $X_x=\mathrm{WReach}_1[G,\sigma](x)$. By maximality, $X_x$ intersects some $X\in\mathcal{M}$; let $y\in X_x\cap X$. We have $y\in \{x\}\cup N(x)$ and $y\in D$, so $x$ is dominated.
\end{proof}

\begin{lemma}[Transversal from optimal solution]
\label{lem:transversal-wreach}
For any minimum dominating set $D^\star$, the set $Y=\bigcup_{d\in D^\star} \mathrm{WReach}_1[G,\sigma](d)$ is a transversal of the hypergraph $\mathcal{H}_\sigma$. Moreover, $|Y|\le r_\sigma\cdot |D^\star|$.
\end{lemma}

\begin{proof}
We show that $Y$ intersects every set in $\mathcal{H}_\sigma$. Take any $x\in V$ and consider $X_x=\mathrm{WReach}_1[G,\sigma](x)$. Since $D^\star$ is a dominating set, there exists $d\in D^\star$ such that $d\in N[x]$. Consider two cases. If $d=x$, then $x\in X_x$ and $x\in\mathrm{WReach}_1[G,\sigma](x)\subseteq Y$, hence $X_x\cap Y\neq\emptyset$. If $d\in N(x)$ and $d\neq x$, we have two possibilities. When $\sigma(d)<\sigma(x)$, then $d\in L_\sigma(x)\subseteq X_x$ and $d\in\mathrm{WReach}_1[G,\sigma](d)\subseteq Y$, so $d\in X_x\cap Y$. When $\sigma(d)>\sigma(x)$, then $x\in L_\sigma(d)\subseteq \mathrm{WReach}_1[G,\sigma](d)\subseteq Y$ and obviously $x\in X_x$, hence again $X_x\cap Y\neq\emptyset$. The size bound follows immediately from $|Y|\le \sum_{d\in D^\star}|\mathrm{WReach}_1[G,\sigma](d)|\le |D^\star|\cdot r_\sigma$.
\end{proof}

\begin{theorem}[Bound on maximum matching size]
\label{thm:wreach-bound}
Let $\mathcal{M}$ be a maximal matching in $\mathcal{H}_\sigma$ and let $D=\bigcup_{X\in\mathcal{M}}X$. Then $D$ is a dominating set and $|D|\le r_\sigma^2\cdot\gamma(G)$.
\end{theorem}

\begin{proof}
From Lemma~\ref{lem:max-dominates-wreach} we know that $D$ is a dominating set. We show the size bound. Let $D^\star$ be a minimum dominating set in $G$, i.e., $|D^\star|=\gamma(G)$. From Lemma~\ref{lem:transversal-wreach}, the set $Y=\bigcup_{d\in D^\star}\mathrm{WReach}_1[G,\sigma](d)$ is a transversal of the hypergraph $\mathcal{H}_\sigma$ and $|Y|\le r_\sigma\cdot\gamma(G)$. 

Since each set in the matching $\mathcal{M}$ is disjoint from the others and each must intersect the transversal $Y$, we obtain $|\mathcal{M}|\le |Y|\le r_\sigma\cdot\gamma(G)$. Each set in $\mathcal{M}$ has size at most $r_\sigma$, hence $|D|=\left|\bigcup_{X\in\mathcal{M}}X\right|\le |\mathcal{M}|\cdot r_\sigma\le r_\sigma\cdot\gamma(G)\cdot r_\sigma=r_\sigma^2\cdot\gamma(G)$.
\end{proof}

\subsection{Result for Out-Stars}

Analogous results hold for the family of out-stars in any orientation.

\begin{lemma}[Domination by maximal matching for stars]
\label{lem:out-domination}
Let $\vec{G}$ be an orientation of graph $G$ and let $\mathcal{M}$ be a maximal matching with respect to inclusion in the hypergraph $\mathcal{H}_{\mathrm{out}}(\vec{G})$. Then the set $D=\bigcup_{S\in\mathcal{M}}S$ is a dominating set in $G$.
\end{lemma}

\begin{proof}
The proof is analogous to the proof of Lemma~\ref{lem:max-dominates-wreach}. For any $x\in V$, if $x\notin D$, then the set $S_{\vec{G}}(x)$ does not belong to $\mathcal{M}$, hence by maximality of $\mathcal{M}$ it must intersect some set $S\in\mathcal{M}$. Let $y\in S_{\vec{G}}(x)\cap S$. Then $y\in\{x\}\cup N^+(x)\cup N^-(x)\subseteq N[x]$ and $y\in D$, so $x$ is dominated.
\end{proof}

\begin{lemma}[Transversal from optimal solution for stars]
\label{lem:out-transversal}
For any dominating set $D^\star$ in graph $G$, the set $Y=\bigcup_{d\in D^\star}S_{\vec{G}}(d)$ is a transversal of the hypergraph $\mathcal{H}_{\mathrm{out}}(\vec{G})$ and $|Y|\le r_{\mathrm{out}}\cdot|D^\star|$.
\end{lemma}

\begin{proof}
For any $x\in V$ consider $S_{\vec{G}}(x)$. There exists $d\in D^\star\cap N[x]$. If $d=x$, then $x\in S_{\vec{G}}(x)\cap S_{\vec{G}}(d)\subseteq S_{\vec{G}}(x)\cap Y$. If edge $\{x,d\}$ is oriented as $(x,d)$, then $d\in N^+(x)\subseteq S_{\vec{G}}(x)$ and $d\in S_{\vec{G}}(d)\subseteq Y$. If it is oriented as $(d,x)$, then $x\in N^+(d)\subseteq S_{\vec{G}}(d)\subseteq Y$ and $x\in S_{\vec{G}}(x)$. In each case $S_{\vec{G}}(x)\cap Y\neq\emptyset$. The size bound follows from $|Y|\le\sum_{d\in D^\star}|S_{\vec{G}}(d)|\le |D^\star|\cdot r_{\mathrm{out}}$.
\end{proof}

\begin{theorem}[Size bound for stars]
\label{thm:out-bound}
Let $\mathcal{M}$ be a maximal matching in $\mathcal{H}_{\mathrm{out}}(\vec{G})$ and let $D=\bigcup_{S\in\mathcal{M}}S$. Then $D$ is a dominating set and $|D|\le r_{\mathrm{out}}^2\cdot\gamma(G)$.
\end{theorem}

\begin{proof}
The proof proceeds identically to the proof of Theorem~\ref{thm:wreach-bound}, using Lemmas~\ref{lem:out-domination} and~\ref{lem:out-transversal}.
\end{proof}

\section{Interface Specification}

The data structure provides the following operations for a dynamic graph $G_t=(V,E_t)$ with bounded degeneracy $\alpha$.

\paragraph{Procedure \textsc{Init}.} Creates an empty structure for a graph on the vertex set $V$. After execution, the graph contains edge set $E_0=\emptyset$, and the returned dominating set $D_0$ is empty. Time complexity is $O(1)$.

\paragraph{Procedure \textsc{Insert-Edge}$(u,v)$.} Adds edge $\{u,v\}$ to graph $G_t$. After execution, the structure maintains a dominating set $D_t$ satisfying $|D_t|\le (\alpha+1)^2\cdot\gamma(G_t)$. Amortized expected time complexity is $O(\alpha^3+\alpha^2\log n)$ against an oblivious adversary.

\paragraph{Procedure \textsc{Delete-Edge}$(u,v)$.} Removes edge $\{u,v\}$ from graph $G_t$. Correctness and complexity guarantees are identical to \textsc{Insert-Edge}.

\paragraph{Procedure \textsc{Query-Dominating-Set}.} Returns the current dominating set $D_t$ in time $\Theta(|D_t|)$.

\section{Data Structure}

The structure consists of four cooperating components.

\paragraph{Orientation module.} We use the dynamic orientation representation for sparse graphs presented by Brodal and Fagerberg in~\cite{BrodalFagerberg99}. For a graph of degeneracy $\alpha$, this representation maintains an orientation $\vec{G}_t$ with guarantee $\Delta^+(\vec{G}_t)=O(\alpha)$. Each edge insertion or deletion operation triggers a deterministically computed sequence of local orientation changes, called \emph{flips}. A flip consists of reversing the direction of a single edge. The amortized cost of adding or deleting an edge in the representation is $O(\alpha+\log n)$ flips. The representation uses $O(n+m)$ memory.

\paragraph{Hypergraph module.} For each vertex $v\in V$ we maintain explicitly the set $S(v)=\{v\}\cup N^+_{\vec{G}_t}(v)$ represented as a list. Changing the orientation of a single edge $(x,y)$ causes modification of at most two sets by adding or removing a single element. The memory of this module is $O(n+m)$.

\paragraph{Matching module.} We maintain a maximal matching $\mathcal{M}_t$ in the current family $\mathcal{H}_{\mathrm{out}}(\vec{G}_t)$ using the algorithm of Assadi and Solomon for dynamic matching in hypergraphs~\cite{AssadiSolomon21}. This algorithm for rank-$r$ hypergraphs guarantees amortized expected time $O(r^2)$ per single hyperedge update, where an update consists of adding or removing an element from a set or complete addition/removal of a set from the family. The algorithm works against an oblivious adversary. Memory is $O(\sum_{v\in V}|S(v)|)=O(n+m)$.

\paragraph{Output module.} For each vertex $x\in V$ we maintain a counter $\mathrm{count}[x]$ equal to the number of sets in $\mathcal{M}_t$ containing $x$. Additionally, we maintain a list of all vertices $x$ such that $\mathrm{count}[x]>0$. This list represents the current dominating set $D_t=\bigcup_{S\in\mathcal{M}_t}S$. Each change in the matching $\mathcal{M}_t$ causes counter updates for at most $r_{\mathrm{out}}=O(\alpha)$ vertices, each costing $O(1)$. The memory of this module is $O(n)$.

Total memory used by the structure is $O(n+m)$.

\section{Operation Implementation}

\subsection{Operation \textsc{Init}}

We initialize the orientation module creating an orientation of the empty graph. For each vertex $v\in V$ we create set $S(v)=\{v\}$. We initialize the matching module with empty matching $\mathcal{M}_0=\emptyset$. All counters $\mathrm{count}[x]$ are set to zero, and the list of vertices in $D_0$ is empty.

\paragraph{Correctness.} The empty set $D_0=\emptyset$ is a dominating set for the empty graph $G_0=(V,\emptyset)$ trivially, since each vertex has an empty neighbor set.

\paragraph{Complexity.} Initialization of the orientation module, empty hyperedge family, empty matching, and zero counters is done in time $O(1)$ assuming preallocation of $O(n)$ memory for header structures.

\subsection{Operation \textsc{Insert-Edge}$(u,v)$}

We call the edge insertion procedure in the orientation module from~\cite{BrodalFagerberg99}. In response, the module performs a deterministically computed sequence of edge flips maintaining the condition $\Delta^+(\vec{G}_t)=O(\alpha)$. For each flip of an edge between vertices $x$ and $y$, we update the corresponding sets $S(x)$ and $S(y)$ by adding or removing a single element. Each such modification is reported to the matching module as a hyperedge update. The matching module appropriately modifies $\mathcal{M}_t$. Each time a set $S$ enters $\mathcal{M}_t$ or leaves $\mathcal{M}_t$, we update counters $\mathrm{count}[x]$ for all $x\in S$ and appropriately modify the list representing $D_t$.

\paragraph{Correctness.} After completion of all operations we have orientation $\vec{G}_t$ with $\Delta^+(\vec{G}_t)\le\alpha$, the current family is $\mathcal{H}_{\mathrm{out}}(\vec{G}_t)$, and matching $\mathcal{M}_t$ is maximal in this family according to guarantees from~\cite{AssadiSolomon21}. From Lemma~\ref{lem:out-domination}, set $D_t$ is dominating. From Theorem~\ref{thm:out-bound} we have $|D_t|\le r_{\mathrm{out}}^2\cdot\gamma(G_t)=(1+\Delta^+(\vec{G}_t))^2\cdot\gamma(G_t)\le (1+\alpha)^2\cdot\gamma(G_t)$.

\paragraph{Complexity.} According to \cite{BrodalFagerberg99}, edge insertion generates amortized $O(\alpha+\log n)$ flips. Each flip causes an update of at most two hyperedges in the matching module. According to \cite{AssadiSolomon21}, each hyperedge update in a rank-$r$ hypergraph costs amortized in expectation $O(r^2)$ against an oblivious adversary. In our case $r=r_{\mathrm{out}}=1+\Delta^+(\vec{G}_t)=O(\alpha)$, hence each hyperedge update costs $O(\alpha^2)$. Total expected amortized cost is $O((\alpha+\log n)\cdot\alpha^2)=O(\alpha^3+\alpha^2\log n)$. Counter updates for each flip cost $O(r_{\mathrm{out}})=O(\alpha)$ and are absorbed in the above estimate.

\subsection{Operation \textsc{Delete-Edge}$(u,v)$}

We call the edge deletion procedure in the orientation module. Further implementation is identical to operation \textsc{Insert-Edge}: the orientation module generates a sequence of flips, each flip causes hyperedge updates, the matching module appropriately modifies $\mathcal{M}_t$, and we update counters and list $D_t$.

\paragraph{Correctness.} The proof is identical to the operation \textsc{Insert-Edge}, using the same lemmas and theorems.

\paragraph{Complexity.} According to \cite{BrodalFagerberg99}, edge deletion also generates amortized $O(\alpha+\log n)$ flips. Further analysis proceeds identically, giving expected amortized cost $O(\alpha^3+\alpha^2\log n)$.

\subsection{Operation \textsc{Query-Dominating-Set}}

We return the list of all vertices $x$ such that $\mathrm{count}[x]>0$. This list is explicitly maintained, so the operation consists of copying it or iterating through it.

\paragraph{Complexity.} Execution time is $\Theta(|D_t|)$, since the list contains exactly $|D_t|$ elements.

\section{Summary}

We presented a fully dynamic data structure for approximate dominating set in graphs of finite degeneracy $\alpha$. The structure consists of four cooperating modules: orientation representation from~\cite{BrodalFagerberg99}, explicit hypergraph representation, dynamic matching from~\cite{AssadiSolomon21}, and an output module maintaining counters. Key combinatorial results show that a maximal matching in the family of backward reachable sets (or equivalently in the family of out-stars) gives a dominating set of size bounded by the square of maximum rank multiplied by the size of the optimal solution. Thanks to the guarantee $\Delta^+(\vec{G}_t)=O(\alpha)$ from~\cite{BrodalFagerberg99}, we obtain approximation factor $(\alpha+1)^2$. Each edge insertion or deletion operation has amortized expected cost $O(\alpha^3+\alpha^2\log n)$ against an oblivious adversary. The structure requires no additional assumptions regarding the relationship between the number of vertices $n$ and degeneracy parameter $\alpha$, works from initialization in constant time, and does not use static algorithms as subroutines.

\bibliographystyle{abbrv}
\begin{thebibliography}{9}

\bibitem{BrodalFagerberg99}
G.~S. Brodal and R.~Fagerberg.
\newblock Dynamic Representations of Sparse Graphs.
\newblock In \emph{Proceedings of WADS 1999}, Lecture Notes in Computer Science, volume 1663, pages 342--351. Springer, 1999.

\bibitem{AssadiSolomon21}
S.~Assadi and S.~Solomon.
\newblock Fully Dynamic Set Cover via Hypergraph Maximal Matching: An Optimal Approximation Through a Local Approach.
\newblock In \emph{Proceedings of the 32nd ACM-SIAM Symposium on Discrete Algorithms (SODA)}, pages 2203--2219, 2021.

\end{thebibliography}

\end{document}