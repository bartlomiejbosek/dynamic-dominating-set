\section{Preliminaries}\label{sec:prelims}

\wninline{Będę chciał mieć definicję co znaczy ``data structure $S(G)$ for a dynamic graph $G$'', aby nie pisać tego 20 razy}

\begin{definition}
	Let $G$ be a graph and $v \in V(G), v' \not\in V(G)$. The operation of \textit{adding a pendant} $vv'$ produces a graph $G'$ such that $V(G') = V(G) \cup \{v'\}$ and $E(G') = E(G) \cup \{vv'\}$.
\end{definition}

\begin{observation} \label{obs:pendants-bnd-exp}
	Let $\G$ be a class of graphs with bounded expansion and let $\G'$ be a class of graphs that can be obtained from $\G$ by adding pendants arbitrarily many times. Then $\G'$ is also of bounded expansion.
\end{observation}

\wninline{Poniższe jest trywialne, ale dla porządku udowodniłem}
\begin{proof}
	Let $\G$ be a class of graphs with bounded expansion and let $f_{\G} : \N \to \R$ be a function such that no $r$-shallow minor $H$ of any $G \in \G$ has $\frac{|E(H)|}{|V(H)|} > f_{\G}(r)$. Let $H'$ be an $r$-shallow minor of some $G' \in \G'$. We have that $H'$ can be obtained from some $r$-shallow minor $H$ of some $G \in \G'$ by repeatedly adding pendants. As adding a pendant increases the number of vertices and edges by one, there exists $c \in \N$ such that $|E(H')| - |E(H)| = c = |V(H')| - |V(H)|$, so $\frac{|E(H')|}{|V(H')|} = \frac{|E(H)| + c}{|V(H)| + c} \le \max(1, \frac{|E(H)|}{|V(H)|}) \le \max(1, f_{\G}(r))$, hence we can set $f_{\G'}(r) \coloneqq \max(1, f_{\G}(r))$ as a function bounding densities of $r$-shallow minors of $\G'$, proving that $\G'$ is of bounded expansion.
\end{proof}


\begin{definition}
	We say that $\F$ is a \textit{type of mappings} if it is a function such that for undirected graphs $H, G$ with colored edges $\F(H, G)$ returns a subset of all mappings from $V(H)$ to $V(G)$.
\end{definition}

For example, homomorphisms, subgraph isomorphisms and induced subgraph isomorphisms are all types of mappings. We shall denote them by $\mathcal{HOM}$, $\mathcal{SUB}$ and $\mathcal{ISUB}$, i.e. $\mathcal{HOM}(H, G)$, $\mathcal{SUB}(H, G)$ and $\mathcal{ISUB}(H, G)$ will denote the set of all homomorphisms, subgraph isomorphisms and induced subgraph isomorphisms, respectively.

\begin{theorem} \label{thm:maintaining-augmentation}
	Placeholder na stwierdzenie, że da się dynamicznie utrzymywać augmentacje. Zapewne chcemy zacytować Theorem 4 z \cite{DvorakTumaArxiv}.
\end{theorem}