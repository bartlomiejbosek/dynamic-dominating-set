\section{Counting and finding small patterns in sparse graphs}\label{sec:dvorak}

Here we recall and extend relevant stuff from Dvo\v{r}\'ak and T\r{u}ma~\cite{DvorakT13}.

\wninline{@Bartek, napiszesz dowód tego lematu? De facto już to zrobiłeś w dom-one i trzeba by go tylko przystosować do języka tutaj}

\newpage
\begin{lemma}
	Let $f$ be a function that takes two arguments --- a graph $G$ and its vertex $v$, such that $f(G, v) \ge 0$ for all valid pairs of arguments.
	
	Suppose that for any fixed weighting function $w : V(G) \to \Z$, there exists a data structure $R_{f, w}(G)$ that for a dynamic graph $G$ is able to report $\sum_{v \in V(G)} f(G, v)w(v)$ after each update.
	
	Also, suppose that either:
	\begin{enumerate}
		\item \label{it:binary-f} $f(G, v) \in \{0, 1\}$ for all valid arguments, or
		\item there exists a data structure $V_{f}(G)$ that for a dynamic graph $G$ and for a given vertex $v$ is able to check if $f(G, v)>0$.
	\end{enumerate}
	Then, there exists a randomized data structure $P_{f}(G)$ with no false positives that is able to report any $v$ such that $f(G, v) > 0$, or that no such vertex exists. If the amortized time and space complexities of $R$ and $V$ per a single update or query are $\Oh(T_R)$, $\Oh(S_R)$, $\Oh(T_V)$ and $\Oh(S_V)$, respectively, then the amortized time and space complexities of $P$ are $\Oh((T_R+T_V) \cdot \log |V(G)| \cdot \log \frac{1}{\eps})$ and $\Oh((S_R+S_V) \cdot \log |V(G)| \cdot \log \frac{1}{\eps})$, where $\eps$ is the probability of a false negative and where we assume that $T_V = S_V = 0$ in the case \ref{it:binary-f}, where $V$ is not required.
\end{lemma}	



\begin{definition}
	We say that $\F$ is a \textit{type of mappings} if it is a function such that for undirected graphs $H, G$ with colored edges $\F(H, G)$ returns a subset of all mappings from $V(H)$ to $V(G)$.
\end{definition}

For example, homomorphisms, subgraph isomorphisms and induced subgraph isomorphisms are all types of mappings. We shall denote them by $\mathcal{HOM}$, $\mathcal{SUB}$ and $\mathcal{ISUB}$, i.e. $\mathcal{HOM}(H, G)$, $\mathcal{SUB}(H, G)$ and $\mathcal{ISUB}(H, G)$ will denote the set of all homomorphisms, subgraph isomorphisms and induced subgraph isomorphisms, respectively.

\begin{definition}
	Let $\F$ be a type of mappings and let $\F(H, G, v, u)$ for $v \in V(H)$ and $u \in V(G)$ be the restriction of $F(H, G)$ to the mappings $\phi$ such that $\phi(v)=u$.
	We say that $\F$ is \textit{reasonable} if and only if it satisfies the following property: Let $H$ and $G$ be undirected graphs with colored edges, where $H$ does not have isolated vertices. Let $v \in V(H), u \in V(G)$, $H'$ be a graph $H$ with added pendant $vv'$ and $G'$ be a graph with added pendant $uu'$, where $vv'$ and $uu'$ have the same color that is not present in $H$ and $G$. Then, $\F(H', G') = \F(H', G', v', u')$ and there is a natural bijection between $\F(H', G')$ and $\F(H, G, v, u)$ given by restricting any $\phi' \in F(H', G')$ to  $V(H)$. 
\end{definition}

\begin{claim}
	Homomorphisms, subgraph isomorphisms and induced subgraph isomorphisms are reasonable types of mappings.
\end{claim}
\begin{proof}
	Let $\F \in \{\mathcal{HOM}, \mathcal{SUB}, \mathcal{ISUB}\}$. Let $\phi \in \F(H', G')$. All three of these types are homomorphisms, so we have that $\phi$ is a homomorphism from $H$ to $G$. As $vv'$ and $uu'$ have equal colors that are different from colors of any other edges in $H'$ and $G'$ we must have that either $\phi(v)=u, \phi(v')=u'$ or $\phi(v)=u', \phi(v')=u$. However, as $v$ is not isolated in $H$, it has an adjacent edge with a different color than $vv'$, so it is not possible that $\phi(v)=u'$. Hence $\F(H', G') = \F(H', G', v', u') = \F(H', G', v, u)$. One can then readily verify that restricting any member of $\F(H', G')$ to $V(H)$ defines a bijection between $\F(H', G')$ and $\F(H, G, v, u)$ for any $\F$ from $\{\mathcal{HOM}, \mathcal{SUB}, \mathcal{ISUB}\}$.
\end{proof}


\begin{lemma} Let us assume that there exists a randomized data structure $S_{H, v, k, \mathcal{F}}(G)$ solving the following problem: Let $H$ be a graph without isolated vertices and with edges colored with $\{1, \ldots, k\}$, $v$ be a vertex of $H$, $G$ be a dynamic graph that belongs to a fixed class of graphs $\mathcal{G}$ of bounded expansion. Let $\mathcal{F}$ be a reasonable type of mappings. Then, the data structure is able to either provide a vertex $u \in V(G)$ such that there exists $\phi$ such that $\phi(v)=u$ and $\phi \in \F(H, G)$, or state that such $\phi$ does not exist. Any update to $G$ is handled in amortized $f_{\mathcal{G}}(|V(H)|, k) \log^{g(|V(H)|, k)}(|V(G)|)$ time complexity (for some monotonic functions $f_{\mathcal{G}}$ and $g$) and the queries never provide false positives, but may provide false negatives with probability at most $\eps$.
	
	Then, there exists a data structure $P_{H, k, \mathcal{F}}(G)$ solving an analogous problem, which provides $\phi$ such that $\phi \in \F(H, G)$, or state that it does not exist. Any update to $G$ is handled in amortized $f'_{\mathcal{G}}(|V(H)|, k) \log^{g'(|V(H)|, k)}(|V(G)|+|V(H)|)$ time complexity for some functions $f_{\mathcal{G}}'$ and $g'$. The queries never provide false negatives, but may provide false negatives with probability at most $|V(H)| \eps$.
	
	%\wninline{Dopisac zalozenie, ze dodanie pendanta do G zachowuje bycie w $\mathcal{G}$.}
\end{lemma}
\begin{proof}
	Let $c=|V(H)|$ and let $V(H) = \{v_1, \ldots, v_c\}$. Let us create a sequence of graphs $H_0, \ldots, H_{c}$, where $H=H_0$ and $H_{i}$ is created from $H_{i-1}$ by adding a $v_ip_i$ pendant colored with the color $k+i$. 
	
	Let $S_i \coloneqq S_{H_i, v_{i+1}, k + i, \mathcal{F}}$ for $i=0, \ldots, c-1$. Any update to $G$ is passed to all of $S_0, \ldots, S_{c-1}$. The data structure $S_i$ will be used to identify the image of $v_{i+1}$ after already fixing images of $v_1, \ldots, v_i$.
	That is, let us assume that we have already identified vertices $u_1, \ldots, u_i \in V(G)$ such that there exists a mapping $\phi \in \F(H, G)$, where $\phi(v_j)=u_j$ for $j \le i$. Let $G_0, G_1, \ldots, G_i$ be the sequence of graphs such that $G=G_0$ and $G_i$ is created from $G_{i-1}$ by adding a $u_iq_i$ pendant colored with the color $k+i$. Let us turn $S_i(G)$ into $S_i(G_i)$ by adding the corresponding $i$ pendants. Let $\G'$ be the class of graphs obtained from $\G$ by adding arbitrarily many pendants. Based on \ref{obs:pendants-bnd-exp} we have that $\G'$ is of bounded expansion and $G_i \in \G'$.
	
	By the assumption that $\F$ is reasonable and by induction on $i$, it is clear that there is a natural bijection between mappings $\phi' \in \F(H_i, G_i)$ and mappings $\phi in \F(H, G)$ such that $\phi(v_i)=u_i$, where that bijection fulfills $\phi'|_{V(H)} = \phi$. As we assumed that there exists $\phi \in \F(H, G)$ such that $\phi(v_i) = u_i$, we conclude that $\F(H_i, G_i)$ is nonempty. Hence, $S_i$ is able to provide a vertex $u_{i+1}$ such that there exists $\phi' \in \F(H_i, G_i)$ where $\phi'(v_{i+1})=u_{i+1}$. Moreover it has to be the case that $u_{i+1} \in V(G)$ (that is, it is not possible that $u_{i+1}=q_j$ for some $j\le i$), so by restricting $\phi'$ to $V(H)$ we get a mapping $\phi$ such that $\phi(v_j)=u_j$ for $j\le i+1$. By repeating this reasoning for $i=0, \ldots, c-1$, we get a full mapping $\phi \in \F(H, G)$, as desired.
	
	Note that we can set $f'_{\G}(|V(H)|, k) = |V(H)|^2 f_{\G'}(2|V(H)|, k+|V(H)|)$ and $g'(|V(H)|, k) = g(2|V(H)|, k+|V(H)|)$ and $|V(G_i)| \le |V(G)| + |V(H)|$. As we make $|V(H)|$ queries to all $S_i$, the tota probability of not succeeding at some step is at most $|V(H)|\eps$.
	
\end{proof}

\begin{definition}
	Let $H$ and $G$ be graphs and $f$ be a function $f : V(H) \times V(G) \to \Z$. Let $\phi : V(H) \to V(G)$ be a mapping. Then, by the \textit{value of} $\phi$, denoted as $\val_f(\phi)$, we denote the following value $\val_f(\phi) = \prod_{v \in V(H)} f(v, \phi(v))$. 
\end{definition}

In \cite{DvorakT13} Dvor{\'{a}}k and Tuma prove the following: 

\begin{theorem}
	Let $H$ be a fixed graph, $\G$ be a class of graphs and $G \in \G$ be a dynamic graph. 
\end{theorem}