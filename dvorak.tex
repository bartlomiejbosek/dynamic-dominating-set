\section{Counting and finding small patterns in sparse graphs}\label{sec:dvorak}

%Here we recall and extend relevant stuff from Dvo\v{r}\'ak and T\r{u}ma~\cite{DvorakT13}.

The main result of the work of Dvo\v{r}\'ak and T\r{u}ma~\cite{DvorakT13} can be expressed as following:

\begin{theorem} \cite{DvorakT13} \label{thm:dvorak-tuma}
	Let $H$ be a fixed graph, $\G$ be a class of graphs and $G \in \G$ be a dynamic graph, where edges of $H$ and $G$ are colored with colors $\{1, \ldots, k\}$. Let $\F$ be either $\mathrm{Hom}$, $\mathrm{Sub}$, or $\mathrm{ISub}$. Then, there exists a~data structure $\mathsf{UnweightedMappings}_{\F, H}(G)$, which is able to determine $|\F(H, G)|$ after each update. If $\G$ is a~class of bounded expansion, then the data structure processes each update in $\Oh_{\G, H, k}(\log^h |G|)$, where $h = {|H| \choose 2} - 1$ and its space complexity is $\Oh_{\G, H, k}(|G|)$. If $\G$ is a nowhere-dense class, then each update takes $\Oh_{\G, H, k}(|G|^\delta)$ time and its space complexity is $\Oh(|G|^{1 + \delta})$.
\end{theorem}

However, they concluded that even though they are able to count appearances of $H$ in $G$ as an induced subgraph (that is, determine $|\ISub(H, G)|$), it is not easy to restore any example of such mapping from their data structure as it involves inclusion-exclusion principle. Despite not written explicitly, it can be expected that the authors would have similar troubles restoring examples of subgraph isomorphisms (as the reduction from counting subgraph isomorphisms to counting homomorphisms requires subtractions as well), though they had likely known how to restore examples of homomorphisms, as that part of the algorithm does not require any subtractions and the restoration of an example can be done through tracing transitions with nonzero contributions in the designed dynamic programming. 

In this work, we are going to resolve these open questions and prove the following:

\begin{theorem} \label{thm:dvorak-tuma-examples}
	Let $H$ be a fixed graph, $\G$ be a class of graphs and $G \in \G$ be a dynamic graph, where edges of $H$ and $G$ are colored with colors $\{1, \ldots, k\}$. Let $\F$ be either $\Hom$, $\Sub$, or $\ISub$. Let $\eps \in (0, 1)$ be a real number. Then, there exists a randomized data structure $\mathsf{MappingExample}_{\F, H, k}(G)$, which after each update is able to report that either $\F(H, G)$ is empty, or provide some $\phi \in \F(H, G)$. If $\G$ is a class of bounded expansion, then the data structure processes each update in $\Oh_{\G, H, k}(\log^{g(|H|)} |G| \log{\frac{1}{\eps}})$ and its space complexity is $\Oh_{\G, H, k}(|G| \log |G| \log {\frac{1}{\eps}})$. If $\G$ is a nowhere-dense class, then each update takes $\Oh_{\G, H, k}(|G|^\delta \log{\frac{1}{\eps}})$ time and its space complexity is $\Oh_{\G, H, k}(|G|^{1 + \delta} \log{\frac{1}{\eps}})$. The data structure never provide false positives, but may fail to provide $\phi$ with probability at most $\eps$.
\end{theorem}

For the algorithms in the following sections we will only need the case $\F = \Hom$, which was the only case that was easy to handle for \cite{DvorakT13} (despite not being stated and proven), but the toolbox that we need to introduce for our main result is of great help in resolving the harder cases of $\F = \Sub$ and $\F = \ISub$ that can be viewed as interesting results of independent interest.

\wninline{@Bartek, napiszesz dowód tego lematu? De facto już to zrobiłeś w dom-one i trzeba by go tylko przystosować do języka tutaj. No i przypadek 2 wymaga jeszcze triku z dzieleniem ważonych przez nieważone aby odzyskać indeks}

The following lemma, which draws an inspiration from \cite{Majewski24} and \cite{Nadara22}, is a handy tool that helps in restorations of examples while given only counting functions.
\begin{lemma} \label{lem:restore-by-sampling}
	Let $f$ be a function that takes two arguments --- a graph $G$ and its vertex $v$, such that $f(G, v) \ge 0$ for all valid pairs of arguments.
	
	Suppose that for any fixed weighting function $w : V(G) \to \Z$, there exists a data structure $\mathsf{WeightedSum}_{f, w}(G)$ that for a dynamic graph $G$ is able to report $\sum_{v \in V(G)} f(G, v)w(v)$ after each update.
	
	Also, suppose that either:
	\begin{enumerate}
		\item \label{it:binary-f} $f(G, v) \in \{0, 1\}$ for all valid arguments, or
		\item \label{it:general-f} there exists a data structure $\mathsf{Verify}_{f}(G)$ that for a dynamic graph $G$ and for a given vertex $v$ is able to check if $f(G, v)>0$.
	\end{enumerate}
	Then, there exists a randomized data structure $\mathsf{RetrieveVertex}_{f}(G)$ with no false positives that is able to report any $v$ such that $f(G, v) > 0$, or that no such vertex exists. If the amortized time complexities per a single update or query and space complexities of $\mathsf{WeightedSum}_{f, w}(G)$ and $\mathsf{Verify}_f(G)$ are $\Oh(T_R)$, $\Oh(S_R)$, $\Oh(T_V)$ and $\Oh(S_V)$, respectively, then the amortized time complexity per a single update and space complexity of $\mathsf{RetrieveVertex}_f(G)$ are $\Oh((T_R+T_V) \cdot \log |G| \cdot \log \frac{1}{\eps})$ and $\Oh((S_R+S_V) \cdot \log |G| \cdot \log \frac{1}{\eps})$, where $\eps$ is the probability of a false negative and where we assume that $T_V = S_V = 0$ in the case \cref{it:binary-f}, where $\mathsf{Verify}$ is not required.
\end{lemma}	

\begin{proof}
Fix an error parameter $\eps\in(0,1)$.
Let $n := |V(G)|$. If $n=0$, then no vertex $v$ with $f(G,v)>0$ exists and the
claim holds trivially, so assume $n\ge 1$.
We assume that $V(G)=\{1,2,\dots,n\}$, i.e., we identify vertices of $G$ with
integers in $\{1,\dots,n\}$.
In particular, we use the same symbol $v$ both for a vertex and for its
identifier in $\{1,\dots,n\}$.
For convenience, in the complexity bounds we write $|G|:=n$.
Set $L := \lfloor \log_2 n\rfloor + 1$ and
\[
  \lambda
    := \max\!\left(1,\,\left\lceil 10 \cdot
       \ln\!\left(\tfrac{1}{\eps}\right)\right\rceil\right).
\]
The maximum with $1$ ensures that $\lambda\ge 1$ even when $\eps$ is close
to~$1$.
The guarantee holds against an oblivious adversary, meaning that the entire
update sequence is fixed independently of the random bits chosen at
initialisation.  In particular, for any fixed time step the resulting graph
state~$G$ is independent of the random choices made below.

\smallskip
\noindent\textbf{Pairwise-independent hashing.}
Let $p$ be a prime such that $8n < p < 16n$
(such a prime exists by Bertrand's postulate applied to the integer $8n>1$),
and fix any such prime $p$ at initialisation (it can be found by testing
integers for primality, which contributes only a one-time additive cost).
For each $j\in\{1,\dots,\lambda\}$ choose $a_j,b_j\in\{0,1,\dots,p-1\}$
independently and uniformly at random and define
$h_j(v) := \bigl(a_j\cdot v+b_j\bigr)\bmod p$.
For every vertex $v$ and $x\in\{0,\dots,p-1\}$, for each choice of $a_j$ there
is exactly one choice of $b_j$ with $h_j(v)=x$, so $h_j(v)$ is uniform.
For any two distinct vertices $u,v$ and any $(x,y)\in\{0,\dots,p-1\}^2$, the
system of congruences $a_j u+b_j\equiv x\pmod p$ and
$a_j v+b_j\equiv y\pmod p$ has a unique solution
$a_j\equiv (x-y)\cdot (u-v)^{-1}\pmod p$ and $b_j\equiv x-a_j u\pmod p$.
Thus $(h_j(u),h_j(v))$ is uniform over $\{0,\dots,p-1\}^2$, and hence each $h_j$
is pairwise-independent.
Each $h_j$ is described by $O(1)$ machine words, and the total seed for all
$\lambda$ functions occupies $O(\lambda)=O(\log\frac{1}{\eps})$ words.
We use the standard word-RAM model with word size $\Omega(\log n)$; since
$p<16n$, each number in $\{0,\dots,p-1\}$ fits in $O(1)$ words and arithmetic
modulo $p$ (hence evaluation of $h_j(v)$ and membership tests $v\in S_{i,j}$)
takes $O(1)$ time.

\smallskip
\noindent\textbf{Random sampling family and the maintained sums.}
For every $i\in\{0,1,\dots,L\}$ and $j\in\{1,2,\dots,\lambda\}$ we define
the threshold
\[
  \tau_i := \left\lfloor \frac{p}{2^{i+1}} \right\rfloor
\]
and the set
\[
  S_{i,j} := \bigl\{v\in V(G) : h_j(v) < \tau_i\bigr\}.
\]
Since $L=\lfloor\log_2 n\rfloor+1$, we have $n<2^{L}$ and thus $2^{L}\le 2n$,
and consequently $2^{L+1}\le 4n$.
Therefore, for every $i\in\{0,\dots,L\}$ we have
$2^{i+1}\le 2^{L+1}\le 4n$.
Since $p>8n$, this implies $p/2^{i+1} > 8n/(4n)=2$ and hence
$\tau_i=\lfloor p/2^{i+1}\rfloor \ge 2$.
Define $p_i := \tau_i/p$.
Because $h_j(v)$ is uniform over $\{0,\dots,p-1\}$,
for every vertex~$v$ we have
\[
  \Pr(v\in S_{i,j})
  = \frac{\tau_i}{p}
  = p_i,
\]
and for every pair of distinct vertices $u,v\in V(G)$
(by the pairwise-independence property of~$h_j$),
\begin{equation}\label{eq:pairwise}
  \Pr(u\in S_{i,j} \;\wedge\; v\in S_{i,j})
  = p_i^{\,2}.
\end{equation}
The random choices are performed once at initialisation and remain fixed
afterwards.  Crucially, membership $v\in S_{i,j}$ can be tested in $O(1)$
time by evaluating $h_j(v)$ from the stored seed; the sets $S_{i,j}$ are
\emph{not} stored explicitly.

For every $(i,j)$ define two weighting functions
$w^{(0)}_{i,j},w^{(1)}_{i,j}\colon V(G)\to\Z$ by
\[
  w^{(0)}_{i,j}(v) :=
  \begin{cases}
    1 & \text{if } v\in S_{i,j},\\
    0 & \text{otherwise,}
  \end{cases}
  \qquad
  w^{(1)}_{i,j}(v) :=
  \begin{cases}
    v & \text{if } v\in S_{i,j},\\
    0 & \text{otherwise.}
  \end{cases}
\]
These are particular weighting functions $w\colon V(G)\to\Z$ as required by the
lemma statement.
Since $S_{i,j}$ is determined by the $O(1)$-word seed of~$h_j$, each weight
$w^{(\ell)}_{i,j}(v)$ can be evaluated in $O(1)$ time without storing
the sets $S_{i,j}$.
Moreover, each weighting function $w^{(\ell)}_{i,j}$ has an $O(1)$-word
representation given by the seed of $h_j$ together with the indices
$(i,\ell)$.
By the convention of the lemma statement, the bounds $T_R,S_R$ for
$\mathsf{WeightedSum}_{f,w}(G)$ account for storing and processing this
$O(1)$-word description of the fixed weighting function~$w$.
In particular, the weighting functions $w^{(0)}_{i,j}$ and $w^{(1)}_{i,j}$ are
fixed after initialisation.

We maintain, for every $(i,j)$, two instances of the promised structure:
$\mathsf{WeightedSum}_{f,\,w^{(0)}_{i,j}}(G)$ and
$\mathsf{WeightedSum}_{f,\,w^{(1)}_{i,j}}(G)$.
In \cref{it:general-f} we additionally maintain one instance of
$\mathsf{Verify}_{f}(G)$.
Every update to~$G$ is relayed to all maintained instances.

\smallskip
\noindent\textbf{Query algorithm.}
Given the current graph~$G$, for all pairs
$(i,j)\in\{0,\dots,L\}\times\{1,\dots,\lambda\}$ in lexicographic order, let
\[
  A_{i,j} := \sum_{v\in V(G)} f(G,v)\,w^{(0)}_{i,j}(v),
  \qquad
  B_{i,j} := \sum_{v\in V(G)} f(G,v)\,w^{(1)}_{i,j}(v),
\]
obtained by querying the respective maintained instances.
If $A_{i,j}=0$, continue to the next pair.
\begin{itemize}
\item In \cref{it:binary-f}, if $A_{i,j}=1$, return the vertex
  $B_{i,j}$ (we verify below that then $B_{i,j}\in\{1,\dots,n\}$); otherwise,
  continue.
\item In \cref{it:general-f}, since $A_{i,j}>0$, compute $q := B_{i,j}/A_{i,j}$.
  If $q$ is not an integer or $q\notin\{1,\dots,n\}$, continue.
  Otherwise, query $\mathsf{Verify}_f(G)$ on the vertex $q$;
  if $f(G,q)>0$, return $q$, and otherwise continue.
\end{itemize}
If no pair leads to returning a vertex, report that no vertex $v$ with
$f(G,v)>0$ exists.

\smallskip
\noindent\textbf{No false positives.}
In \cref{it:binary-f}, suppose the algorithm returns $B_{i,j}$ for
some $(i,j)$.
By definition of $w^{(0)}_{i,j}$ and the assumption
$f(G,v)\in\{0,1\}$, we have
\[
  A_{i,j}=\sum_{v\in S_{i,j}} f(G,v)=1,
\]
hence there exists exactly one vertex $x\in S_{i,j}$ with $f(G,x)=1$, and
for all $v\in S_{i,j}\setminus\{x\}$ we have $f(G,v)=0$.
Therefore
\[
  B_{i,j}=\sum_{v\in S_{i,j}} f(G,v)\,v=x,
\]
so the returned vertex is $B_{i,j}=x$ and satisfies
$f(G,x)=1>0$.

In \cref{it:general-f}, the algorithm returns a vertex $q$ only after
$\mathsf{Verify}_f(G)$ has confirmed that $f(G,q)>0$; hence every
returned vertex satisfies $f(G,v)>0$. In particular, even if $q$ happens to be
an integer in $\{1,\dots,n\}$ but corresponds to a vertex $y$ with $f(G,y)=0$,
the verification step prevents returning~$y$.

Thus, in both cases the data structure has no false positives.

\smallskip
\noindent\textbf{Probability of a false negative.}
Fix a state of the graph~$G$ and let
\[
  X := \{v\in V(G)\mid f(G,v)>0\}
  \qquad\text{and}\qquad
  m := |X|.
\]
Formally, we consider a fixed time step (that is, a graph state obtained after
a fixed number of updates); since the update sequence is fixed independently of
the random bits of the construction, the resulting graph~$G$ (and hence $X$)
is determined independently of the hash functions $h_1,\dots,h_\lambda$.
Equivalently, we may condition on this fixed graph state~$G$ without changing
the distribution of the random hash functions.
If $m=0$, then $f(G,v)=0$ for all $v\in V(G)$ and consequently $A_{i,j}=0$ for
every $(i,j)$, so the query algorithm correctly reports that no such vertex
exists. Hence assume $m\ge 1$.

Set $i^\star := \lfloor \log_2 m\rfloor + 1$.
Since $m\le n$ and $L = \lfloor\log_2 n\rfloor + 1$, we have
$i^\star \le L$, so $i^\star\in\{0,1,\dots,L\}$.
Note that for $m=1$ we have $i^\star=1$.
Moreover, $2^{i^\star-1}\le m < 2^{i^\star}$: the left inequality holds
because $i^\star - 1 = \lfloor \log_2 m\rfloor$ and
$2^{\lfloor\log_2 m\rfloor}\le m$; the right inequality holds because
$m < 2^{\lfloor\log_2 m\rfloor+1} = 2^{i^\star}$.

Define $\mu := m\cdot p_{i^\star} = m\cdot \tau_{i^\star}/p$.
Since $\tau_{i^\star}=\lfloor p/2^{i^\star+1}\rfloor \le p/2^{i^\star+1}$,
we have
\[
  \mu \le m\cdot 2^{-i^\star-1}
  < 2^{i^\star}\cdot 2^{-i^\star-1}
  = 1/2.
\]
Moreover, $\tau_{i^\star} > p/2^{i^\star+1}-1$, so
\[
  \mu > m\cdot 2^{-i^\star-1} - \frac{m}{p}
  \ge 2^{i^\star-1}\cdot 2^{-i^\star-1} - \frac{m}{p}
  = \frac14 - \frac{m}{p}.
\]
$p>8n\ge 8m$, so $m/p < 1/8$, and therefore
$\mu\in(1/8,\,1/2)$.

Fix any $j\in\{1,\dots,\lambda\}$.
For each $v\in X$ let $I_v := \mathbf{1}[v\in S_{i^\star,j}]$ be the
indicator random variable and set $Y := \sum_{v\in X} I_v
= |S_{i^\star,j}\cap X|$.
We have $\Pr(I_v = 1) = p_{i^\star}$ for each $v\in X$, and by the
pairwise-independence property~\eqref{eq:pairwise}, for any two distinct
vertices $u,v\in X$:
\[
  \Pr(I_u = 1 \;\wedge\; I_v = 1) = p_{i^\star}^{\,2}.
\]

We lower-bound $\Pr(Y=1)$.
For each $v\in X$, the event $\{I_v=1\}$ is the disjoint union of the events
\[
  E_v := \Bigl\{I_v = 1 \;\wedge\; \bigwedge_{u\in X\setminus\{v\}} I_u=0\Bigr\}
  \qquad\text{and}\qquad
  F_v := \bigl\{I_v=1\;\wedge\;\exists\,u\in X\setminus\{v\}:\;I_u=1\bigr\},
\]
hence $\Pr(E_v)=\Pr(I_v=1)-\Pr(F_v)$.
Observe that the event $\{Y=1\}$ is exactly the disjoint union
$\biguplus_{v\in X} E_v$, and hence $\Pr(Y=1)=\sum_{v\in X}\Pr(E_v)$.
By the union bound we have
\[
  \Pr(F_v) \le \sum_{u\in X\setminus\{v\}} \Pr(I_v=1\;\wedge\;I_u=1).
\]
Therefore:
\begin{align*}
  \Pr(Y=1)
  &= \sum_{v\in X} \Pr\!\Bigl(I_v=1 \;\wedge\;
     \textstyle\bigwedge_{u\in X\setminus\{v\}} I_u=0\Bigr)\\
  &= \sum_{v\in X}\Bigl[\Pr(I_v=1)
     - \Pr\!\bigl(I_v=1\;\wedge\;\exists\,u\in X\setminus\{v\}:\;I_u=1\bigr)
     \Bigr]\\
  &\ge \sum_{v\in X}\Bigl[p_{i^\star}
     - \sum_{u\in X\setminus\{v\}} \Pr(I_v=1 \;\wedge\; I_u=1)\Bigr]\\
  &= m\,p_{i^\star} - m(m-1)\,p_{i^\star}^{\,2}
  \;=\; \mu - \tfrac{m-1}{m}\,\mu^2
  \;\ge\; \mu - \mu^2
  \;=\; \mu(1-\mu).
\end{align*}
The function $g(\mu):=\mu(1-\mu)$ satisfies $g'(\mu)=1-2\mu > 0$ for
$\mu<1/2$, hence $g$ is increasing on $[0,\,1/2)$.
Therefore $g(\mu) > g(1/8) = 7/64 > 1/10$ (since $7\cdot 10 > 64$), and so
\[
  \Pr\bigl(|S_{i^\star,j}\cap X|=1\bigr)=\Pr(Y=1)\ge \tfrac{1}{10}.
\]

The hash functions $h_1,\dots,h_\lambda$ are sampled independently.
For each fixed $j$, the set $S_{i^\star,j}$ (and hence the event
$\{|S_{i^\star,j}\cap X|\neq 1\}$) is determined solely by~$h_j$.
Therefore, the events $\{|S_{i^\star,j}\cap X|\neq 1\}$ are independent across
distinct values of~$j$.
Consequently
\[
  \Pr\Bigl(\forall\, j\in\{1,\dots,\lambda\}\colon\;
    |S_{i^\star,j}\cap X|\neq 1\Bigr)
  \le \left(1-\tfrac{1}{10}\right)^\lambda
  \le \exp\!\left(-\tfrac{\lambda}{10}\right)
  \le \exp\!\left(-\ln\!\left(\tfrac{1}{\eps}\right)\right)
  = \eps,
\]
where we used $(1-x)\le e^{-x}$ for $x\ge 0$ and the definition
$\lambda\ge 10\ln(1/\eps)$.

We now argue that if for some~$j$ we have $|S_{i^\star,j}\cap X|=1$, then
the query algorithm returns a vertex from~$X$ no later than when it processes
the pair $(i^\star,j)$.
Let~$x$ be the unique element of $S_{i^\star,j}\cap X$.
By definition of $X$, we have $f(G,x)>0$.
\begin{itemize}
\item In \cref{it:binary-f}, $f(G,x)=1$ and $f(G,v)=0$ for all
  $v\in S_{i^\star,j}\setminus\{x\}$
  (since every such $v$ satisfies $v\notin X$, hence $f(G,v)=0$ by
  definition of~$X$), so
  $A_{i^\star,j}=\sum_{v\in S_{i^\star,j}}f(G,v)=f(G,x)=1$ and
  $B_{i^\star,j}=\sum_{v\in S_{i^\star,j}}f(G,v)\,v=x$.
  The query algorithm returns~$x$ when it encounters this pair.
\item In \cref{it:general-f}, since $x$ is the only element of
  $S_{i^\star,j}$ with $f(G,\cdot)>0$ (every other $v\in S_{i^\star,j}$
  satisfies $v\notin X$ and hence $f(G,v)=0$), we have
  $A_{i^\star,j}=f(G,x)>0$ and $B_{i^\star,j}=f(G,x)\cdot x$.
  Therefore $q = B_{i^\star,j}/A_{i^\star,j}=x\in\{1,\dots,n\}$,
  which is an integer.
  The algorithm then queries $\mathsf{Verify}_f(G)$ on~$x=q$,
  which confirms $f(G,x)>0$, and returns~$x$.
\end{itemize}
Moreover, if the algorithm returns a vertex for any \emph{earlier} pair
$(i',j')$ --- not necessarily with $i'=i^\star$ --- the returned vertex still
satisfies $f(G,v)>0$ by the no-false-positives property, so this is also
not a false negative.
Hence a false negative occurs only when the algorithm exhausts all pairs
without returning, which requires
$|S_{i^\star,j}\cap X|\neq 1$ for every $j\in\{1,\dots,\lambda\}$, an event
of probability at most~$\eps$.

\smallskip
\noindent\textbf{Complexity.}
We maintain $2(L+1)\lambda$ instances of~$\mathsf{WeightedSum}_{f,w}(G)$.
In \cref{it:general-f} we additionally maintain one instance of
$\mathsf{Verify}_f(G)$.
To combine amortized bounds across maintained instances, we sum their total
costs over an arbitrary operation sequence and divide by the number of outer
updates/queries.
Since $L=\lfloor\log_2 n\rfloor+1$, we have $L+1=\Theta(1+\log n)$.
Here and below, $\log$ denotes a logarithm to an arbitrary fixed base, so
changing the base affects only constant factors in $\Oh(\cdot)$.
In particular, for $n\ge 2$ we have $L+1=\Theta(\log n)=\Theta(\log|G|)$, and
therefore $(L+1)\lambda=\Theta(\log|G|\cdot\log\frac{1}{\eps})$.
The sampling of the hash functions and the initialisation of the maintained
instances is performed once and contributes only a one-time additive cost; we
focus on the amortized costs of subsequent updates and queries.

\emph{Update cost.}
Each update to~$G$ is relayed to all $2(L+1)\lambda$ instances of
$\mathsf{WeightedSum}$ and to the single instance of $\mathsf{Verify}$
(in \cref{it:general-f}).
Thus, every update to~$G$ triggers $2(L+1)\lambda$ update operations in
$\mathsf{WeightedSum}$ instances and one update operation in
$\mathsf{Verify}$ (in \cref{it:general-f}). Summing the amortized bounds over
all maintained instances, the total amortized update time is therefore
\[
  \Oh\!\bigl(2T_R\cdot (L+1)\lambda + T_V\bigr)
  \;\subseteq\;
  \Oh\!\bigl((T_R+T_V)\cdot (L+1)\lambda\bigr)
  \;=\;
  \Oh\!\left((T_R+T_V)\cdot \log |G|\cdot
    \log\tfrac{1}{\eps}\right).
\]

\emph{Query cost.}
The query algorithm iterates over at most $(L+1)\lambda$ pairs.
For each pair it performs two queries to $\mathsf{WeightedSum}_{f,w}(G)$
and at most one query to $\mathsf{Verify}_f(G)$, each costing $\Oh(T_R)$
and $\Oh(T_V)$, respectively.
Thus, each query to $\mathsf{RetrieveVertex}_f(G)$ performs at most
$2(L+1)\lambda$ queries to $\mathsf{WeightedSum}$ instances and at most
$(L+1)\lambda$ queries to $\mathsf{Verify}$ (in \cref{it:general-f}).
Summing the amortized bounds, this gives an amortized query time of
\[
  \Oh\!\bigl(2T_R\cdot (L+1)\lambda + T_V\cdot (L+1)\lambda\bigr)
  \;\subseteq\;
  \Oh\!\bigl((T_R+T_V)\cdot (L+1)\lambda\bigr).
\]
Equivalently, over any sequence with $N_u$ updates and $N_q$ queries,
$\mathsf{Verify}$ receives $N_u$ updates and at most $N_q\cdot (L+1)\lambda$
queries, so its amortized contribution per outer operation is
$\Oh((L+1)\lambda\cdot T_V)$.
In \cref{it:general-f}, for each processed pair with $A_{i,j}>0$ we additionally
compute $q = B_{i,j}/A_{i,j}$ and test integrality; this is a division of
integers whose bit-length is bounded by the output size of
$\mathsf{WeightedSum}$ and is accounted for in~$\Oh(T_R)$ by the conventions of
the lemma statement.
The total query time is therefore
$\Oh\!\left((T_R+T_V)\cdot\log |G|\cdot\log\frac{1}{\eps}\right)$,
matching the stated bound.

\emph{Space.}
The $2(L+1)\lambda$ instances of $\mathsf{WeightedSum}$ use
$\Oh(S_R\cdot\log |G|\cdot\log\frac{1}{\eps})$ space.
The single instance of $\mathsf{Verify}$ uses $\Oh(S_V)$ space.
The hash seeds for $h_1,\dots,h_\lambda$ occupy $\Oh(\lambda) =
\Oh(\log\frac{1}{\eps})$ machine words and contribute an additional additive
$\Oh(\log\frac{1}{\eps})$ term.
The sets $S_{i,j}$ are \emph{not} stored explicitly: membership
$v\in S_{i,j}$ is tested in $\Oh(1)$ time by evaluating $h_j(v)$ from the
stored seed.
The total space is therefore
$\Oh\!\left(S_R\cdot\log|G|\cdot\log\frac{1}{\eps} + S_V +
  \log\frac{1}{\eps}\right)$.
Since the $O(1)$-word description of the fixed weighting function is accounted
for in $S_R$, we have $S_R\ge 1$.
In particular, for $|G|\ge 2$ (so $\log|G|\ge 1$) the additive
$\Oh(\log\frac{1}{\eps})$ term is subsumed and the space is
$\Oh\!\left((S_R+S_V)\cdot\log|G|\cdot\log\frac{1}{\eps}\right)$.

\end{proof}


Before proceeding we are going to define a useful property of mappings that is shared by homomorphisms, subgraph isomorphisms and induced subgraph isomorphisms.
\begin{definition}
	Let $\F$ be a type of mappings and let $\F(H, G, v, u)$ for $v \in V(H)$ and $u \in V(G)$ be the restriction of $F(H, G)$ to the mappings $\phi$ such that $\phi(v)=u$.
	We say that $\F$ is \textit{reasonable} if and only if it satisfies the following property: Let $H$ and $G$ be undirected graphs with colored edges. Let $v \in V(H), u \in V(G)$, $H'$ be a graph $H$ with added pendants $vv', vv''$ and $G'$ be a graph $G$ with added pendants $uu', uu''$, where $\col(vv') = \col(uu')$, $\col(vv'') = \col(uu'')$ and $\col(vv'), \col(vv'') \not\in \col(E(H)) \cup \col(E(G))$. Then, $\F(H', G') = \F(H', G', v, u) = \F(H', G', v', u') = F(H', G', v'', u'')$ and there is a natural bijection between $\F(H', G')$ and $\F(H, G, v, u)$ given by restricting any $\phi' \in F(H', G')$ to  $V(H)$. 
\end{definition}

\begin{claim}
	Homomorphisms, subgraph isomorphisms and induced subgraph isomorphisms are reasonable types of mappings.
\end{claim}
\begin{proof}
	Let $\F \in \{\Hom, \Sub, \ISub\}$. Let $\phi \in \F(H', G')$. All three of these types are homomorphisms, so we have that $\phi$ is a homomorphism from $H$ to $G$. As $v$ and $u$ are the only vertices adjacent to edges of colors $\col(vv')$ and $\col(vv'')$ in $H'$ and $G'$, respectively, we must have that either $\phi(v)=u$, and consequently $\phi(v') = u'$ and $\phi(v'') = u''$. Hence $\F(H', G') = \F(H', G', v, u) = \F(H', G', v', u') = \F(H', G', v'', u'') = $. One can then readily verify that restricting any member of $\F(H', G')$ to $V(H)$ defines a bijection between $\F(H', G')$ and $\F(H, G, v, u)$ for any $\F$ from $\{\Hom, \Sub, \ISub\}$.
\end{proof}

The proof of \cref{thm:dvorak-tuma-examples} will consist of two steps. First, we are going to design a data structure $\mathsf{MapVertex}$ that will identify a possible image of a single vertex of $H$. Then, we are going to use it to identify images of all vertices of $H$ one by one.

Let us focus on designing $\mathsf{MapVertex}$ first. The idea behind designing it is to apply \cref{lem:restore-by-sampling} for a function $f$, where $f(G, u)$ denotes the number of mappings from $\F$ that map $v$ to $u$. However, using that tool requires introducing weighting functions.

First, let us introduce a weighted value of a mapping:

\begin{definition} \label{def:mapping-value}
	Let $H$ and $G$ be graphs and $w$ be a weighting function $w : V(H) \times V(G) \to \Z$. Let $\phi : V(H) \to V(G)$ be a mapping. Then, by the \textit{value of} $\phi$, denoted as $\val_w(\phi)$, we denote the following value $\val_w(\phi) = \prod_{v \in V(H)} f(v, \phi(v))$. Consequently, for a type of mappings $\F$, we are going to define $\val_w(\F(H, G))$ as $\sum_{\phi \in \F(H, G)} \val_w(\phi)$.
\end{definition}

With that definition, we shall introduce a weighted version of \cref{thm:dvorak-tuma} that will help us in applying \cref{lem:restore-by-sampling}. 

\begin{theorem} \label{lem:weighted-dvorak}
	Let $H$ be a fixed graph, $\G$ be a class of graphs and $G \in \G$ be a dynamic graph, where edges of $H$ and $G$ are colored with colors $\{1, \ldots, k\}$ and $w$ be a weighting function $w : V(H) \times V(G) \to \Z$. Let $\F$ be either $\Hom$, $\Sub$, or $\ISub$. Then, there exists a data structure $\mathsf{WeightedCount}_{\F, H, w}(G)$, which is able to determine $\val_w(\F(H, G))$ after each update. If $\G$ is a class of bounded expansion, then the data structure processes each update in $\Oh_{\G, H, k}(\log^h |G|)$, where $h = {|H| \choose 2} - 1$ and its space complexity is $\Oh_{\G, H, k}(|G|)$. If $\G$ is a nowhere-dense class, then each update takes $\Oh_{\G, H, k}(|G|^\delta)$ time and its space complexity is $\Oh_{\G, H, k}(|G|^{1 + \delta})$.
\end{theorem}

On a high level, \cref{thm:dvorak-tuma} is proven in a way that counting induced subgraph isomorphisms is reduced to counting subgraph isomorphisms, which is reduced to counting homomorphisms, which is reduced to counting homomorphisms in a special case, where both $H$ and $G$ are additionally directed, but $H$ is a so called \textit{elder graph}. The last case is solved through a dynamic programming routine. As can be expected, incorporating weights into a dynamic programming routine can be easily done and having that, weighted versions are consequently easily pulled back to homomorphisms, subgraph isomorphisms and induced subgraph isomorphism in that order. We defer more detailed proof of \cref{lem:weighted-dvorak} to \cref{sec:app-weighted-dvorak}.

With \cref{lem:weighted-dvorak}, we are now able to design the promised $\mathsf{MapVertex}$ data structure:

\begin{lemma} \label{lem:map-vertex}
	Let $H$ be a fixed graph, $\G$ be a class of graphs, $G \in \G$ be a dynamic graph, where edges of $H$ and $G$ are colored with colors $\{1, \ldots, k\}$, $\eps \in (0, 1)$ be a real number, let $v \in V(H)$ and let $\F \in \{\Hom, \Sub, \ISub\}$. Then, there exists a randomized data structure $\mathsf{MapVertex}_{\F, H, v, k}$ that is able to either provide a vertex $u \in V(G)$ such that there exists $\phi$ such that $\phi(v)=u$ and $\phi \in \F(H, G)$, or state that such $\phi$ does not exist. If $\G$ has bounded expansion, then the amortized time complexity of handling an update to $G$ is $\Oh_{\G, H, k}(\log^{g(|H|)}|G| \log \frac{1}{\eps})$ for some function $g$ and its space complexity is $\Oh_{\G, H, k}(|G| \log |G| \log \frac{1}{\eps})$, whereas if $\G$ is nowhere-dense, then the amortized time complexity of an update to $G$ is $\Oh_{\G, H, k}(|G|^{\delta} \log \frac{1}{\eps})$ and its space complexity is $\Oh_{\G, H, k}(|G|^{1 + \delta} \log \frac{1}{\eps})$. $\mathsf{MapVertex}_{\F, H, v, k}$ never provides false positives, but may provide a false negative with probability at most $\eps$.
\end{lemma}
\begin{proof}
	This proof will be an application of \cref{lem:restore-by-sampling}. In order to apply it, we are going to provide required data structures $\mathsf{WeightedSum}_{f, w}(G)$ and $\mathsf{Verify}_f(G)$ such that the resulting $\mathsf{RetrieveVertex}_f(G)$ will exactly be our desired $\mathsf{MapVertex}_{\F, H, v, k}(G)$.
		
	Let $f(G, u)$ be a function that returns the number of $\phi$ such that $\phi(v) = u$ and $\phi \in \F(H, G)$ (we clearly have that $f(G, u) \ge 0$) and let $w$ be some fixed weighting function $w : V(G) \to \Z$. Let us define $w' : V(H) \times V(G) \to \Z$ in the following way:
	
	\[
	w'(x, y) =
	\begin{cases}
		w(y) & \text{if } x = v, \\
		1  & \text{otherwise}
	\end{cases}
	\]
	
	For $w'$ defined in such a way, it is easy to see that $\val_{w'}(\F(H, G)) = \sum_{u \in V(G)} f(G, u) w(u)$.
	Hence, we can set $\mathsf{WeightedSum}_{f, w}(G) \coloneqq \mathsf{WeightedCount}_{\F, H, w'}(G)$.
	
	As for designing $\mathsf{Verify}_f(G)$, let us first define $H'$, which is obtained by adding to $H$ two pendants $vv'$ and $vv''$ with colors $k+1$ and $k+2$ and for $u \in V(G)$ let us define $G_u$ by adding to $G$ two pendants $uu'$ and $uu''$ with colors $k+1$ and $k+2$ as well. As $\F$ is a reasonable type of mappings, we have that $\F(H', G_u) = \F(H, G, v, u)$. Hence, checking if $f(G, u)>0$ is equivalent to checking if $|\F(H', G_u)| > 0$. Consequently, $\mathsf{Verify}_f(G)$ can be implemented in the following way. Let us initialize $\mathsf{UnweightedMappings}_{\F, H'}$ on a set of vertices of size $V(G) \cup \{p_1, p_2\}$. All updates to $G$ are relayed to that structure. Whenever a query about a vertex $u$ comes to $\mathsf{Verify}_f(G)$, it adds edges $up_1$ and $up_2$ with colors $k+1$ and $k+2$ to $\mathsf{UnweightedMappings}_{\F, H'}(G)$ in order to transform $G$ to $G_u$ and answers that $f(G, u)>0$ if and only if $|\F(H', G_u)| > 0$. After answering the query, the edges $up_1$ and $up_2$ are removed.
	
	If $\G'$ is the class of graphs obtained by adding pendants to $\G$, then $G_u \in \G'$. If $\G$ has bounded expansion, then the time complexities of both handling an update to $G$ or answering a single query by $\mathsf{Verify}_f(G)$ are $\Oh_{\G', H', k+2}(\log^{g(|H'|)} |G'|) = \Oh_{\G, H, k}(\log^{g'(|H|)} |G|)$, while its space complexity is $\Oh_{\G, H, k}(|G|)$. Whereas, if $\G$ is nowhere-dense, then the time complexities of handling an update to $G$ or answering a single query are $\Oh_{\G, H, k}(|G|^\delta)$, while its space complexity is $\Oh_{\G, H, k}(|G|^{1 + \delta})$.
	
	Applying \cref{lem:restore-by-sampling} with the provided functions $f, w$ and supplied data structures $\mathsf{WeightedSum}_{f, w}(G)$ and $\mathsf{Verify}_f(G)$ results in obtaining $\mathsf{RetrieveVertex}_f(G)$ that serves as $\mathsf{MapVertex}_{\F, H, v, k}(G)$ with the desired properties.
	
\end{proof}

%\wninline{Napisały mi się lematy w takiej kolejności w jakiej mi się napisały, ale myślę, że może lepiej jednak je pozamieniać kolejnością, aby nie mieć przydługiego założenia o istnieniu MapVertex z abstrakcyjną złożonością i dopiero potem dowodzić jego istnienia, tylko jednak najpierw udowodnić jego istnienie, a potem już operować na konkretach?}
\wninline{Zdecydowałem się dodać po dwa pendanty, a nie po jednym, aby nie było problemów z izolowanymi wierzchołkami}
%\begin{lemma} \label{lem:vertices-to-mappings}
%% Then, the data structure is able to either provide a vertex $u \in V(G)$ such that there exists $\phi$ such that $\phi(v)=u$ and $\phi \in \F(H, G)$, or state that such $\phi$ does not exist. Any update to $G$ is handled in amortized $\Oh_{\G, H, k}(\log^{g(|H|, k)}(|G|))$ time complexity (for some monotonic function $g$) and the queries never provide false positives, but may provide false negatives with probability at most $\eps$.
%Let $H$ be a fixed graph, $\G$ be a class of graphs, $G \in \G$ be a dynamic graph and let $\F \in \{\Hom, \Sub, \ISub\}$. Then, there exists a data structure $\mathsf{MappingExample}_{\F, H, k}(G)$ that provides $\phi$ such that $\phi \in \F(H, G)$, or state that it does not exist. Any update to $G$ is handled in amortized $\Oh_{\G, H, k}(\log^{g(|H|)}(|G|))$ time complexity for some function $g'$. The queries never provide false negatives, but may provide false negatives with probability at most $|H| \eps$.
%	
%\end{lemma}

We are now ready to conclude the proof of the \cref{thm:dvorak-tuma-examples}.

\begin{proof} [Proof of \cref{thm:dvorak-tuma-examples}]
	Let $V(H) = \{v_1, \ldots, v_{|H|}\}$. Let us create a sequence of graphs $H_0, \ldots, H_{|H|}$, where $H=H_0$ and $H_{i}$ is created from $H_{i-1}$ by adding $v_ip_{2i-1}$ and $v_ip_{2i}$ pendants colored with the colors $k+2i-1$ and $k+2i$, respectively. 
	
	Let $\mathsf{MapVertex}_i \coloneqq \mathsf{MapVertex}_{\F, H_i, v_{i+1}, k + 2i}$ for $i=0, \ldots, |H|-1$. Any update to $G$ is passed to all of $\mathsf{MapVertex}_0, \ldots, \mathsf{MapVertex}_{|H|-1}$. The data structure $\mathsf{MapVertex}_i$ will be used to identify the image of $v_{i+1}$ after already fixing images of $v_1, \ldots, v_i$.
	That is, let us assume that we have already identified vertices $u_1, \ldots, u_i \in V(G)$ such that there exists a mapping $\phi \in \F(H, G)$, where $\phi(v_j)=u_j$ for $j \le i$. Let $G_0, G_1, \ldots, G_i$ be the sequence of graphs such that $G=G_0$ and $G_i$ is created from $G_{i-1}$ by adding $u_iq_{2i-1}$ and $u_iq_{2i}$ pendants colored with the colors $k+2i-1$ and $k+2i$, respectively (for that reason $\mathsf{MapVertex}_i$ should be initialized on $|G|+2i$ vertices rather than on $|G|$). Let us turn $\mathsf{MapVertex}_i(G)$ into $\mathsf{MapVertex}_i(G_i)$ by adding the corresponding $2i$ pendants. Let $\G'$ be the class of graphs obtained from $\G$ by adding arbitrarily many pendants. Based on \cref{obs:pendants-bnd-exp} we have that $\G'$ is of bounded expansion and $G_i \in \G'$.
	
	By the assumption that $\F$ is reasonable and by induction on $i$, it is clear that there is a natural bijection between mappings $\phi' \in \F(H_i, G_i)$ and mappings $\phi \in \F(H, G)$ such that $\phi(v_i)=u_i$, where that bijection fulfills $\phi'|_{V(H)} = \phi$. As we assumed that there exists $\phi \in \F(H, G)$ such that $\phi(v_i) = u_i$, we conclude that $\F(H_i, G_i)$ is nonempty. Hence, $\mathsf{MapVertex}_i$ is able to provide a vertex $u_{i+1}$ such that there exists $\phi' \in \F(H_i, G_i)$ where $\phi'(v_{i+1})=u_{i+1}$. Moreover it has to be the case that $u_{i+1} \in V(G)$ (that is, it is not possible that $u_{i+1}=q_j$ for some $j\le i$), so by restricting $\phi'$ to $V(H)$ we get a mapping $\phi$ such that $\phi(v_j)=u_j$ for $j\le i+1$. By repeating this reasoning for $i=0, \ldots, |H|-1$, we get a full mapping $\phi \in \F(H, G)$, as desired.
	
	Note that we can set $g'(|H|, k) = g(3|H|, k+2|H|)$ and $|G_i| \le |G| + 2|H| = \Oh_{H}(|G|)$. As we make $|H|$ queries to all $\mathsf{MapVertex}_i$ structures in total, the total probability of not succeeding at some step is at most $|H|\eps$.
	
\end{proof}


We remark that as $|\F(H, G)| \le |G|^{|H|}$ and $\val_w(\phi) \le |G|$ for any considered $\F$, $w$ and $\phi$, all integers involved in the computations are of size at most $\Oh(|H|)$ words, hence for a constant $H$, any arithmetic computations can be performed in constant time as well. 


\wninline{Dopisać możliwości rozszerzenia o struktury relacyjne}
%We also remark that \cref{thm:dvorak-tuma-examples} can be easily extended to the setting of considering mappings between relational structures provided that the class of Gaifman graphs of corresponding relational structures is still either of bounded expansion or nowhere-dense. The argument is described the Section 4 of \cite{DvorakTumaArxiv} and it carries seamlessly to our statement as Dvo\v{r}\'ak and T\r{u}ma show that any 